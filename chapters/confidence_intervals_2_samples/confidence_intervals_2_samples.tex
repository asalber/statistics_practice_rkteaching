% Author: Alfredo Sánchez Alberca (asalber@ceu.es)

\chapter{Confidence intervals for comparing two populations}\label{cha:confidence-intervals-2}

\section{Solved exercises}
\begin{enumerate}[leftmargin=*] 
\item  In order to see whether an advertising campaign has increased the sales of a drug, a sample of 8 pharmacies were
drawn from a city.
In each pharmacy the monthly sales of the drug before and after the campaign was recorded in the following table.
\begin{center}
\begin{tabular}{lrrrrrrrr}
\toprule
Before & 147 & 163 & 121 & 205 & 132 & 190 & 176 & 147 \\
After & 150 & 171 & 132 & 208 & 141 & 184 & 182 & 145\\ 
\bottomrule
\end{tabular}
\end{center}

\begin{enumerate}
\item Create a data set with the variables \variable{before} and \variable{after} and enter the data.

\item Compute the mean of monthly sales, before and after the campaign. 
Are the meas different?
Have the campaign increased the sales of the drug?
\begin{indication}
\begin{enumerate}
\item Select the menu \menu{Teaching > Descriptive statistics > Statistics}.
\item In the dialog displayed insert the variables \variable{before} and \variable{after} in the field
\field{Variables}.
\item In the \mtab{Basic statistics} tab check the box \option{Mean} and click the button \button{Submit}.
\end{enumerate}
\end{indication}

\item Compute the confidence intervals for the mean of the difference between the monthly sales after and before with
confidence levels $0.05$ and $0.01$.
\begin{indication}
\begin{enumerate}
\item Select the menu \menu{Teaching > Parametric tests > Means > t-test for comparing the means of two paired
populations}.
\item In the dialog displayed insert the variable \variable{before} in the field \field{Compare mean of population} and
the variable \variable{after} in the field \field{With mean of population}.
\item In the \mtab{Test options} enter $0.95$ in the field \field{Confidence level} and click the button
\button{Submit}.
\item Repeat the previous steps but entering $0.99$ in the field \field{Confidence level}.
\end{enumerate}
\end{indication}

\item Can we affirm that the advertising campaign has increased the drug sales significantly? 
Can we conclude the same if we change the sales after the campaign of the two last pharmacies putting 190 instead of 182
and 165 instead of 145?
What happens to the confidence intervals?
\begin{indication}
\begin{enumerate}
\item Select the menu \menu{Teaching > Parametric tests > Means > t-test for comparing the means of two paired
populations}.
\item In the dialog displayed insert the variable \variable{before} in the field \field{Compare} and the variable
\variable{after} in the field \field{With}.
\item In the \mtab{Test options} enter $0.95$ in the field \field{Confidence level} and click the button
\button{Submit}.
\end{enumerate}
We can conclude that there are significant differences between the means, with the confidence level set, if the
confidence interval doesn't contain zero. 
\end{indication}
\end{enumerate}


\item A diary company receive milk from two farms $X$ and $Y$.
To analyze the quality of milk, the milk fat have been measure for two samples of milk, one from each farm.
The results are in the table below. 
\[
\begin{array}{ll|ll}
\multicolumn{2}{c|}{X} & \multicolumn{2}{c}{Y} \\
\hline
0.34 & 0.34 & 0.28 & 0.29 \\
0.32 & 0.35 & 0.30 & 0.32 \\
0.33 & 0.33 & 0.32 & 0.31 \\
0.32 & 0.32 & 0.29 & 0.29 \\
0.33 & 0.30 & 0.31 & 0.32 \\
0.31 & 0.32 & 0.29 & 0.31 \\
 &  & 0.33 & 0.32 \\
 &  & 0.32 & 0.33 \\
\end{array}
\]

\begin{enumerate}
\item Create a data set with the variables \variable{fat} and \variable{farm} and enter the data of the sample.
\item Compute the confidence interval for the difference between the milk fat means of farms $X$ and $Y$.  
\begin{indication}
\begin{enumerate}
\item Select the menu \menu{Teaching > Parametric tests > Means > t-test for comparing the means of two independent
populations}.
\item In the dialog displayed insert the \variable{fat} in the field \field{Compare mean of} and the variable
\variable{farm} in the field \field{According to}.
\item In the frame \field{Populations to compare} insert the value \variable{X} in the field \field{Compare} and the
value \variable{Y} in the field \field{With}.
\item In the \mtab{Test options} enter $0.95$ in the field \field{Confidence level} and click the button
\button{Submit}.
\end{enumerate}
\end{indication}

\item Can we conclude that the difference between the milk fat means of the farms is significant? 
Which farm has milk with more fat?
How much more fat has the milk of $X$ farm than the milk of $Y$ farm? 
\begin{indication}
We can conclude that there are significant differences between the means, with the confidence level set, if the
confidence interval doesn't contain zero. 
\end{indication}
\end{enumerate}


\item In a survey performed by a university about the use of the library, a random sample of 34 students has been asked
whether they go to the library at least once a week.
The answers and the gender of the students are shown below. 
\begin{center}
\resizebox{\linewidth}{!}{
\begin{tabular}{lccccccccccccccccc}
\toprule
Answer & no & yes & no & no & no & yes & no & yes & yes & yes & yes & no & yes & no & yes & no & no \\
Gender & m & f & f & m & m & m & f & f & f & f & m & m & f & m & f & m & m \\
\hline
Answer & no & yes & yes & yes & no & no & yes & no & no & yes & yes & no & no & yes & no & yes & no \\
Gender & f & m & f & f & f & m & f & m & m & f & f & m & m & f & f & f & m\\
\bottomrule
\end{tabular}
}
\end{center}

\begin{enumerate}
\item Create a data set with the variables \variable{answer} and \variable{gender}.
\item Compute the confidence interval for the difference between the proportions of females and males that use the
library at least once a week.
\begin{indication}
\begin{enumerate}
\item Select the menu \menu{Teaching > Parametric tests > Proportions > Test for comparing two proportions}.
\item In the dialog displayed insert the variable \variable{answer} in the field \field{Compare} and 
the variable \variable{gender} in the field \field{According to}.
\item Insert the value \variable{yes} in the field \field{Proportion of}.
\item In the frame \field{Populations to compare} insert the value \variable{f} in the field \field{Compare population} and the value \variable{m} in the field \field{With population}.
\item In the \mtab{Test options} enter $0.95$ in the field \field{Confidence level} and click the button
\button{Submit}.
\end{enumerate}
We can conclude that there are significant differences between the proportions, with the confidence level set, if the
confidence interval doesn't contain zero. 
\end{indication}
\end{enumerate}

\item In a course there are two groups of students, one in the morning and the other in the afternoon.
In the morning group 55 students of 80 passed, while in the afternoon group 32 students of 90 passed.
Are there significant differences between the percentages of students that passed in the morning and in the afternoon?
Can we conclude that the timetable is the cause of the differences?
Justify the answer. 
\begin{indication}
\begin{enumerate}
\item Select the menu \menu{Teaching > Parametric tests > Proportions > Test for comparing two proportions}.
\item In the dialog displayed check the box \option{Manual entry of frequencies}, enter $55$ in the field \field{Sample
frequency 1}, enter 80 in the field \field{Sample size 1}, enter 32 in the field \field{Sample frequency 2}, enter 90 in
the field \field{Sample size 2} and click the button \button{Submit}.
\end{enumerate}
\end{indication}
\end{enumerate}


\section{Proposed exercises}
\begin{enumerate}[leftmargin=*] 
\item  In a study to determine the relation between the physical exercise and the level cholesterol in blood, a sample
of 11 persons was drawn. 
The participants cholesterol level (in mg/dl) before and after doing a program of physical exercises is shown below. 
\begin{center}
\begin{tabular}{lrrrrrrrrrrr}
\toprule
Cholesterol before & 182 & 232 & 191 & 200 & 148 & 249 & 276 & 213 & 241 & 280 & 262 \\
Cholesterol after  & 198 & 210 & 194 & 220 & 138 & 220 & 219 & 161 & 210 & 213 & 226 \\
\bottomrule
\end{tabular}
\end{center}

\begin{enumerate}
\item Compute the 95\% confidence interval for the mean of the difference of cholesterol level before and after doing
exercise.
\item Compute the 99\% confidence interval for the mean of the difference of cholesterol level before and after doing
exercise.
\item According to the previous confidence intervals, what can you conclude about the effect of physical exercise on
cholesterol?
\end {enumerate}

\item In a pool performed in two hospital of a city patients were asked whether they were satisfied with the treatment. 
In the first hospital 200 patients were asked and 140 answered yes, while in the second hospital 300 patients were asked
and 180 answered yes. 
\begin{enumerate}
\item Compute the confidence interval for the difference of proportions of patients satisfied in each hospital.
\item Is there a significant difference, with a significance level $0.01$, between the proportions of people satisfied
in each hospital?
\end{enumerate}

\item The data set \variable{neonates} of the package \variable{rk.Teaching}, contains information about a
sample of 320 newborns that meet the normal gestation time in a hospital during one year.
Do the following operations:
\begin{enumerate}
\item Compute the confidence interval for the difference between the weight means of neonates of non-smoker and smoker
mothers.
How much increase on average the weight of a newborn of a non-smoker mother compared to that of a smoker mother? 

\item Considering only the sample of neonates of mothers that didn't smoke during the pregnancy, compute the confidence
interval for the difference between the weight means of neonates of mothers that didn't smoke before the pregnancy and
mothers that smoked before the pregnancy.
Has the fact of smoking before the pregnancy a significant influence on the weight of neonates? 

\item Compute the confidence interval for the mean of the difference between the APGAR scores at 1 minute and at 5
minutes.
How is the evolution of neonates during the first minutes of life?
\item If neonates with an APGAR score at 1 minute less than or equal to 3 are considered depressed, compute the 90\%
confidence interval for the difference between the proportions of depressed neonates with a smoker mother and with
a non-smoker mother.
Have the fact of smoking during the pregnancy a significant influence in the depression of neonates?
\item Have the age of the mother a significant influence in the depression of neonates?
\end{enumerate}
\end{enumerate}
