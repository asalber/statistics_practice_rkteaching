% Author: Alfredo Sánchez Alberca (asalber@ceu.es)

\chapter{Discrete Random Variables}


\section{Solved exercises}
\begin{enumerate}[leftmargin=*] 

\item Let $X$ be the variable that measures the number of heads got after tossing 10 coins, following a Binomial probability distribution model $B(10,0.5)$.
%Para ver de manera experimental the distribución de probability de $X$ se realiza un random experiment que consiste en lanzar varias
% veces las 10 monedas and anotar el número de heads obtenido en cada lanzamiento. 
\begin{enumerate}
% \item Lanzar las 10 monedas 1000 veces and calcular las frequencies relativas de las heads obtenidas and el diagrama de
% barras asociado.
% \begin{indication}Para generar los lanzamientos de monedas:
% \begin{enumerate}
% \item Select the menu \menu{Teaching>Simulaciones>Lanzamiento de monedas}.
% \item In the dialog shown, enter 10 in the field \field{Número de monedas}, 1000 in the field
% \field{Número de lanzamientos}, enter un nombre para el data set and click the
% button aceptar\button{Submit}.
% \end{enumerate}
% Para calcular las frequencies relativas:
% \begin{enumerate}
% \item Select the menu \menu{Teaching>Distribución de frequencies>Tabla de frequencies}.
% \item In the dialog shown, select como variable a tabular the variable \variable{sum} and hacer clic
% en el button \button{Submit}.
% \end{enumerate}
% Para dibujar el diagrama de barras:
% \begin{enumerate}
% \item Select the menu \menu{Teaching>Gráficos>Diagrama de barras}.
% \item In the dialog shown select the variable \variable{sum}.
% \item En the solapa \menu{Opciones de las barras} marcar the opción \option{Frequencies relativas} and click the
% button \button{Submit}.
% \end{enumerate}}
% \end{indication}

\item Compute the probability distribution of $X$. 
%y compararla con the distribución de frequencies relativas del apartado anterior.
\begin{indication}
\begin{enumerate}
\item Select the menu \menu{Teaching > Distributions > Discretes > Binomial > Probabilities}.
\item In the dialog shown, enter \command{0,1,2,3,4,5,6,7,8,9,10} in the field \field{Values of the variable},
enter \command{10} in the field \field{Number of repetitions}, \command{0.5} in the field \field{Probability of success}, and click the button \button{Submit}.
\end{enumerate}
\end{indication}

\item Plot the graph of the probability function of $X$.
% and compararla con el diagrama de barras de frequencies relativas del primer apartado.
\begin{indication}
\begin{enumerate}
\item Select the menu \menu{Teaching > Distributions > Discretes > Binomial > Probability graph}.
\item In the dialog shown, enter \command{10} in the field \field{Number of repetitions},
\command{0.5} in the field \field{Probability of success} and click the button \button{Submit}.
\end{enumerate}
\end{indication}

\item Plot the graph of the distribution function of $X$.
\begin{indication}
\begin{enumerate}
\item Select the menu \menu{Teaching > Distributions > Discretes > Binomial > Probability graph}.
\item In the dialog shown, enter \command{10} in the field \field{Number of repetitions}, \command{0.5} in the field
\field{Probability of success}, check the box \option{Distribution function} and click the button
\button{Submit}.
\end{enumerate}
\end{indication}

\item Compute the probability of getting 7 heads.
\begin{indication}
\begin{enumerate}
\item Select the menu \menu{Teaching > Distributions > Discretes > Binomial > Probabilities}.
\item In the dialog shown, enter \command{7} in the field \field{Values of the variable},
enter \command{10} in the field \field{Number of repetitions}, \command{0.5} in the field \field{Probability of success}, and click the button \button{Submit}.
\end{enumerate}
\end{indication}

\item Compute the probability of getting less than 4 heads.
\begin{indication}
\begin{enumerate}
\item Select the menu \menu{Teaching > Distributions > Discretes > Binomial > Probabilities}.
\item In the dialog shown, enter \command{4} in the field \field{Values of the variable}, \command{10} in the field
\field{Number of repetitions}, \command{0.5} in the field \field{Probability of success}, check the box \option{Cumulative probabilities} and click the button \button{Submit}.
\end{enumerate}
\end{indication}

\item Compute the probability of getting more than 5 heads.
\begin{indication}
\begin{enumerate}
\item Select the menu \menu{Teaching > Distributions > Discretes > Binomial > Probabilities}.
\item In the dialog shown, enter \command{5} in the field \field{Values of the variable}, \command{10} in the field
\field{Number of repetitions}, \command{0.5} in the field \field{Probability of success}, check the box \option{Cumulative probabilities}, check the box \option{Upper} in the field \field{Accumulation tail} and click the button \button{Submit}.
\end{enumerate}
\end{indication}

\item Compute the probability of getting two or more heads and less than 9.
\begin{indication}
\begin{enumerate}
\item Select the menu \menu{Teaching > Distributions > Discretes > Binomial > Probabilities}.
\item In the dialog shown, enter the values \command{1,8} in the field \field{Values of the variable}, \command{10} in the field \field{Number of repetitions}, \command{0.5} in the field \field{Probability of success}, check the box \option{Cumulative probabilities} and click the button \button{Submit}.
\end{enumerate}
The probability $P(2\leq X<9)$ is the difference between the probabilities obtained $P(X<9)=P(X\leq 8)$ and $P(X<2)=P(X\leq 1)$.
\end{indication}
\end{enumerate}


\item The number of births in a city $X$ follows a Poisson probability distribution model with mean 6 births a day.
\begin{enumerate}
\item Plot the graph of the probability function of $X$.
\begin{indication}
\begin{enumerate}
\item Select the menu \menu{Teaching > Distributions > Discretes > Poisson > Probability graph}.
\item In the dialog shown, enter the value \command{6} in the field \field{Mean} and click the button \button{Submit}.
\end{enumerate}
\end{indication}

\item Plot the graph of the distribution function of $X$.
\begin{indication}
\begin{enumerate}
\item Select the menu \menu{Teaching > Distributions > Discretes > Poisson > Probability graph}.
\item In the dialog shown, enter the value \command{6} in the field \field{Mean}, check the box \option{Distribution function} and click the button \button{Submit}.
\end{enumerate}
\end{indication}

\item Compute the probability that there is 1 birth a random day.  
\begin{indication}
\begin{enumerate}
\item Select the menu \menu{Teaching > Distributions > Discretes > Poisson > Probabilities}.
\item In the dialog shown, enter \command{1} in the field \field{Values of the variable}, enter 6 in the field \field{Mean}, and click the button \button{Submit}.
\end{enumerate}
\end{indication}

\item Compute the probability that there are less than 6 births a random day.
\begin{indication}
\begin{enumerate}
\item Select the menu \menu{Teaching > Distributions > Discretes > Poisson > Probabilities}.
\item In the dialog shown, enter \command{5} in the field \field{Values of the variable}, enter \command{6} in the field
\field{Mean}, check the box \option{Cumulative probabilities} and click the button \button{Submit}.
\end{enumerate}
\end{indication}

\item Compute the probability that there are 4 or more births a random day. 
\begin{indication}
\begin{enumerate}
\item Select the menu \menu{Teaching > Distributions > Discretes > Poisson > Probabilities}.
\item In the dialog shown, enter \command{3} in the field \field{Values of the variable}, enter \command{6} in the field
\field{Mean}, check the box \option{Cumulative probabilities}, select the option \option{Upper} in the field
\field{Accumulation tail} and click the button \button{Submit}.
\end{enumerate}
\end{indication}

\item Compute the probability that there are between 4 and 8 births, both included, a random day. 
\begin{indication}
\begin{enumerate}
\item Select the menu \menu{Teaching > Distributions > Discretes > Poisson > Probabilities}.
\item In the dialog shown, enter \command{3,8} in the field \field{Values of the variable}, enter \command{6} in the field \field{Mean}, check the box \option{Cumulative probabilities} and click the button \button{Submit}.
\end{enumerate}
The probability $P(4\leq X\leq 8)$ is the difference between the probabilities obtained $P(X\leq 8)$ and $P(X<4)=P(X\leq 3)$.
\end{indication}

\item Compute the probability that there are more than 30 and less than 40 births in a week. 
\begin{indication}
\begin{enumerate}
\item Select the menu \menu{Teaching > Distributions > Discretes > Poisson > Probabilities}.
\item In the dialog shown, enter \command{30,39} in the field \field{Values of the variable}, enter \command{42} in the field \field{Mean}, check the box \option{Cumulative probabilities} and click the button \button{Submit}.
\end{enumerate}
The probability $P(30< X< 40)$ is the difference between the probabilities obtained $P(X<40)=P(X\leq 39)$ and $P(X\leq 30)$.
\end{indication}
\end{enumerate}


\item The law of rare events asserts that the Binomial probability distribution model $B(n,p)$, tends to the Poisson probability distribution model $P(np)$ when $n$ tends to $\infty$ and $p$ tends to $0$. 
In particular, the Poisson model is a good approximation of the Binomial model for $n\geq 30$ and $p\leq 0.1$.
To check this law, 
\begin{enumerate}
\item Compute the probability distribution of the Binomial model $B(30,0.1)$.
\begin{indication}
\begin{enumerate}
\item Select the menu \menu{Teaching > Distributions > Discretes > Binomial > Probabilities}.
\item In the dialog shown, enter \command{0,1,2,3,4,5,6,7,8,9,10} in the field \field{Values of the variable}, enter \command{30} in the field \field{Number of repetitions}, \command{0.1} in the field \field{Probability of success} and click the button \button{Submit}.
\end{enumerate}
\end{indication}

\item Compute the probability distribution of the Poisson model $P(3)$ and compare it with the Binomial distribution $B(30,0.1)$.
\begin{indication}
\begin{enumerate}
\item Select the menu \menu{Teaching > Distributions > Discretes > Poisson > Probabilities}.
\item In the dialog shown, enter \command{0,1,2,3,4,5,6,7,8,9,10} in the field \field{Values of the variable}, enter \command{3} in the field \field{Mean} and click the button \button{Submit}.
\end{enumerate}
\end{indication}

\item Compute the probability distribution of the Binomial model $B(100,0.03)$ and compare it with the Poisson distribution $P(3)$.
Are these distributions models more similar than the previous ones? 
\begin{indication}
\begin{enumerate}
\item Select the menu \menu{Teaching > Distributions > Discretes > Binomial > Probabilities}.
\item In the dialog shown, enter \command{0,1,2,3,4,5,6,7,8,9,10} in the field \field{Values of the variable}, enter \command{100} in the field \field{Number of repetitions}, \command{0.03} in the field \field{Probability of success} and click the button \button{Submit}.
\end{enumerate}
\end{indication}

\item Plot the graphs of the probability functions of the previous models. 
Increase number of repetitions and decrease the probability of success in the Binomial model and observe how the probabilities of the Binomial and the Poisson models are more similar.   
\begin{indication}
\begin{enumerate}
\item Select the menu \menu{Teaching > Simulations > Law of rare events}.
\item In the dialog shown, enter \command{30} in the field \option{n} and enter \command{0.1} in the field \option{p}.
\item Then increase the value of \option{n} up to \command{100} and decrease the value of \option{p} down to \command{0.03}.
\end{enumerate}
\end{indication}
\end{enumerate}
\end{enumerate}


\section{Proposed exercises}
\begin{enumerate}[leftmargin=*]
\item What is the probability of getting between 40 and 60 heads, both included, after tossing 100 coins?

\item The chance of being cured with a treatment is $0.85$. 
If we apply the treatment to 6 patients,
\begin{enumerate}
\item Plot the graph of the probability function of the number of patients cured.  
\item What is the probability that half of them are cured?
\item What is the probability that a least 4 of them are cured?
\end{enumerate}

\item The probability of having an adverse reaction to a vaccine is $0.001$. 
If 2000 persons are vaccinated, what is the probability of having some adverse reaction?

\item The average number of calls per minute that arrive to a telephone switchboard is 120. 
\begin{enumerate}
\item What is the probability of receiving less than 4 calls in 2 seconds?
\item What is the probability of receiving at least 3 calls in 3 seconds?
\end{enumerate}
\end{enumerate}







