% Author: Alfredo Sánchez Alberca (asalber@ceu.es)

\chapter{Frequency distributions and charts}\label{cha:freqency-distributions}

\section{Solved exercises}
\begin{enumerate}[leftmargin=*]

\item The number of children in a sample of 25 is
\begin{center}
1, 2, 4, 2, 2, 2, 3, 2, 1, 1, 0, 2, 2, 0, 2, 2, 1, 2, 2, 3, 1, 2, 2, 1, 2.
\end{center}
Do the following operations:
\begin{enumerate}
\item Create a data frame with the variable \variable{children} and enter the data.

\item Create the frequency table.
\begin{indication}
\begin{enumerate}
\item Select the menu \menuc{Teaching > Frequency distribution > Frequency tabulation} .
\item In the dialog displayed, select the variable \variable{children} in the field \field{Variable to tabulate} and
click the button \button{Send}.
\end{enumerate}
\end{indication}

\item Create the absolute frequency bar chart.
\begin{indication}
\begin{enumerate}
\item Select the menu \menu{Teaching > Charts > Bar chart}.
\item In the dialog displayed, select the variable \variable{children} in the field \field{Variable} and click
the button \button{Send}.
\end{enumerate}
\end{indication}

\item Create also the relative frequency, cumulative absolute frequency and cumulative relative frequency bar charts,
with their respective polygons.
\begin{indication}Follow the steps above checking, in the \option{Bar options} tab, the box
\option{Relative frequencies} for the relative frequencies bar chart, the box \option{Cumulative
frequencies} for the cumulative absolute bar chart, and both of them for the cumulative relative frequency bar chart. 
Check the box \option{Polygon}, to plot the corresponding polygon. 
\end{indication}
\end{enumerate}

\item The number of people treated in the emergency service of a hospital every day of November was
\begin{center}
15 \quad 23 \quad 12 \quad 10 \quad 28 \quad 7 \quad 12 \quad 17 \quad 20 \quad 21 \quad 18 \quad 13 \quad 11 \quad 12 \quad 26 \\
30 \quad 6 \quad 16 \quad 19 \quad 22 \quad 14 \quad 17 \quad 21 \quad 28 \quad 9 \quad 16 \quad 13 \quad 11 \quad 16 \quad 20
\end{center}
Do the following operations: 
\begin{enumerate}
\item Create a data frame with the variable \variable{emergencies} and enter the data.

\item Create the box plot. Are there some outlier? In that case, remove the outliers and proceed with the next part.
\begin{indication}
\begin{enumerate}
\item Select the menu \menu{Teaching > charts > Box plot}.
\item In the dialog displayed select the variable \variable{emergencies} in the field \field{Variables} and
click the button \button{Send}.
\item In the output windows with the box plot identify the outliers.
\item In the data frame edition tab, remove the rows with the outliers right-clicking the row header and selecting
\menu{Delete this row}.
\end{enumerate}
\end{indication}

\item Create the frequency table grouping data into 5 classes.
\begin{indication}
\begin{enumerate}
\item Select the menu \menu{Teaching > Frequency distribution > Frequency tabulation}.
\item In the dialog displayed select the variable \variable{emergencies}.
\item In the \option{Classes} tab check the box \option{Grouping intervals}, check the option \option{Number of
intervals}, enter the desired number of intervals in the field \field{Suggested intervals} and
click the button \button{Send}.
\end{enumerate}
\end{indication}

\item Create the absolute frequencies histogram.
\begin{indication}
\begin{enumerate}
\item Select the menu \menu{Teaching > charts > Histogram}.
\item In the dialog displayed select the variable \variable{emergencies} in the field \field{Variable}.
\item In the \option{Classes} tab, check the box \option{Grouping intervals}, check the box
\option{Number of intervals}, enter the desired number of intervals in the field \field{Suggested intervals} and click
the button \button{Send}.
\end{enumerate}
\end{indication}

\item Create also the relative frequency, cumulative absolute frequency and cumulative relative frequency
histograms, with their respective polygons.
\begin{indication}Follow the steps above checking, in the \option{Histogram options},
the box \option{Relative frequencies} for the relative frequency histogram, the box \option{Cumulative frequencies} for
the cumulative absolute frequency histogram, and both of them for the cumulative relative frequency histogram.
Check the box \option{Polygon} to plot the corresponding polygon. 
\end{indication}
\end{enumerate}

\item The blood type of a sample of 30 persons are:
\begin{center}
A, B, B, A, AB, 0, 0, A, B, B, A, A, A, A, AB,\\
A, A, A, B, 0, B, B, B, A, A, A, 0, A, AB, 0. 
\end{center}
Do the following operations:
\begin{enumerate}
\item Create a data frame with the variable \variable{blood.type} and enter the data.

\item Create the frequency table.
\begin{indication}
\begin{enumerate}
\item Select the menu \menu{Teaching > Frequency distribution > Frequency tabulation}.
\item In the dialog displayed, select the variable \variable{blood.type} in the field
\field{Variable to tabulate} and click the button \button{Send}.
\end{enumerate}
\end{indication}

\item Create the pie chart.
\begin{indication}
\begin{enumerate}
\item Select the menu \menu{Teaching > charts > Pie chart}.
\item In the dialog displayed, select the variable \variable{blood.type} in the field \field{Variables} and click the
button \button{Send}.
\end{enumerate}
\end{indication}
\end{enumerate}

\item The age and the marital status of a sample of 27 persons are:
\begin{center}
\begin{tabular}{|l|rrrrrrrrr|}
\hline
Marital status & \multicolumn{9}{c|}{Age}\\
\hline
Single    & 31 & 45 & 35 & 65 & 21 & 38 & 62 & 22 & 31 \\
Married     & 62 & 39 & 62 & 59 & 21 & 62 &    &    &    \\
Widow(er)      & 80 & 68 & 65 & 40 & 78 & 69 & 75 &    &    \\
Divorced & 31 & 65 & 59 & 49 & 65 &    &    &    &    \\
\hline
\end{tabular}
\end{center}

Do the following operations:
\begin{enumerate}
\item Create a data frame with the variables \variable{marital.status} and \variable{age} and enter the data.
\item Create the frequency table of the variable \variable{age} for every marital status.
\begin{indication}
\begin{enumerate}
\item Select the menu \menu{Teaching > Frequency distribution > Frequency tabulation}.
\item In the dialog displayed, select the variable \variable{age} in the field \field{Variable to
tabulate}, check the box \option{Tabulate by groups}, select the variable \variable{marital.status} in the field
\field{Grouping variable(s)} and click the button \button{Send}.
\end{enumerate}
\end{indication}

\item Create the box plots of age for every marital status. Are there outliers? Which group have more spread in ages?
\begin{indication}
\begin{enumerate}
\item Select the menu \menu{Teaching > charts > Box plot}.
\item In the dialog displayed, select the variable \variable{age} in the field \field{Variables},
check the box \option{Plot by groups}, select the variable \variable{marital.status} in the field
\field{Grouping variable} and click the button \button{Send}.
\end{enumerate}
\end{indication}
\end{enumerate}

\end{enumerate}


\section{Proposed exercises}
\begin{enumerate}[leftmargin=*]

\item  El número de lesiones padecidas durante una temporada por cada jugador de un equipo de fútbol fue el siguiente:
\begin{center}
0, 1, 2, 1, 3, 0, 1, 0, 1, 2, 0, 1, 1, 1, 2, 0, 1, 3, 2, 1, 2, 1, 0, 1
\end{center}

Do the following operations:
\begin{enumerate}
\item Construir la frequency table.
\item Dibujar el diagrama de barras de las relative frequencies and de relative frequencies acumuladas.
\item Dibujar el pie chart.
\end{enumerate}

\item Para realizar un estudio sobre la estatura de los estudiantes universitarios, seleccionamos, mediante un proceso
de muestreo aleatorio, una muestra de 30 estudiantes, obteniendo los siguientes resultados (medidos en centímetros):
\begin{center}
179, 173, 181, 170, 158, 174, 172, 166, 194, 185,\\
162, 187, 198, 177, 178, 165, 154, 188, 166, 171,\\
175, 182, 167, 169, 172, 186, 172, 176, 168, 187.
\end{center}

Do the following operations:
\begin{enumerate}
\item Dibujar el histograma de las absolute frequencies agrupando desde 150 a 200 en clases de amplitud 10.
\item Dibujar el box plot. ¿Existe algún outlier?.
\end{enumerate}

\item El conjunto de datos \variable{neonatos} del paquete \variable{rk.Teaching}, contiene información sobre una
muestra de 320 recién nacidos en un hospital durante un año que cumplieron el tiempo normal de gestación. 
Do the following operations:
\begin{enumerate}
\item Construir la frequency table de la puntuación Apgar al minuto de nacer. 
Si se considera que una puntuación Apgar de 3 o menos indica que el neonato está deprimido, ¿qué porcentaje de niños está deprimido en la muestra?
\item Comparar las distribuciones de frecuencias de las puntuaciones Apgar al minuto de nacer según si la madre es mayor
o menor de 20 años.
¿En qué grupo hay más neonatos deprimidos?
\item Construir la frequency table para el peso de los neonatos, agrupando en clases de amplitud $0.5$ desde el
$2$ hasta el $4.5$. ¿En qué intervalo de peso hay más niños?
\item Comparar la distribución de relative frequencies del peso de los neonatos según si la madre fuma o no. Si se
considera como peso bajo un peso menor de $2.5$ kg, ¿En qué grupo hay un mayor porcentaje de niños con peso bajo?
\item Si en los recién nacidos se considera como peso bajo un peso menor de $2.5$ kg, calcular la prevalencia del bajo
peso de recién nacidos en el grupo de madres fumadoras and en el de no fumadoras. 
\item Calcular el riesgo relativo de que un recién nacido tenga bajo peso cuando la madre fuma, frente a cuando la madre
no fuma. 
\item Construir el diagrama de barras de la puntuación Apgar al minuto. ¿Qué puntuación Apgar es la más frecuente? 
\item Construir el diagrama de relative frequencies acumuladas de la puntuación Apgar al minuto. ¿Por debajo de que puntuación estarán la mitad de los niños?
\item Comparar mediante diagramas de barras de relative frequencies las distribuciones de las puntuaciones Apgar al
minuto según si la madre ha fumado o no durante el embarazo. ¿Qué se puede concluir?
\item Construir el histograma de pesos, agrupando en clases de amplitud $0.5$ desde el $2$ hasta el $4.5$. ¿En qué
intervalo de peso hay más niños?
\item Comparar la distribución de relative frequencies del peso de los neonatos según si la madre fuma o no. ¿En qué
grupo se aprecia menor peso de los niños de la muestra?
\item Comparar la distribución de relative frequencies del peso de los neonatos según si la madre fumaba o no antes del
embarazo. ¿Qué se puede concluir?
\item Construir el diagrama de caja and bigotes del peso. ¿Entre qué valores se considera que el peso de un neonato es
normal? ¿Existen datos atípicos?
\item Comparar el box plot and bigotes del peso, según si la madre fumó o no durante el embarazo and si era mayor o
no de 20 años. ¿En qué grupo el peso tiene más dispersión central? ¿En qué grupo pesan menos los niños de la
muestra?
\item Comparar el box plot de la puntuación Apgar al minuto and a los cinco minutos. ¿En qué variable hay más
dispersión central?
\end{enumerate}  

\end{enumerate}
