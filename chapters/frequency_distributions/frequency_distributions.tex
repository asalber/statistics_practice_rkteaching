% Author: Alfredo Sánchez Alberca (asalber@ceu.es)

\chapter{Frequency distributions and charts}\label{cha:freqency-distributions}

\section{Solved exercises}
\begin{enumerate}[leftmargin=*]

\item The number of children in a sample of 25 families is
\begin{center}
1, 2, 4, 2, 2, 2, 3, 2, 1, 1, 0, 2, 2, 0, 2, 2, 1, 2, 2, 3, 1, 2, 2, 1, 2.
\end{center}
Do the following operations:
\begin{enumerate}
\item Create a data set with the variable \variable{children} and enter the data.

\item Create the frequency table.
\begin{indication}
\begin{enumerate}
\item Select the menu \menu{Teaching > Frequency distribution > Frequency table} .
\item In the dialog displayed, select the variable \variable{children} in the field \field{Variable to tabulate} and
click the button \button{Submit}.
\end{enumerate}
\end{indication}

\item Create the absolute frequency bar chart.
\begin{indication}
\begin{enumerate}
\item Select the menu \menu{Teaching > Charts > Bar chart}.
\item In the dialog displayed, select the variable \variable{children} in the field \field{Variable} and click
the button \button{Submit}.
\end{enumerate}
\end{indication}

\item Create also the relative frequency, cumulative absolute frequency and cumulative relative frequency bar charts,
with their respective polygons.
\begin{indication}Follow the steps above checking, in the \option{Bar options} tab, the box
\option{Relative frequencies} for the relative frequencies bar chart, the box \option{Cumulative
frequencies} for the cumulative absolute bar chart, and both of them for the cumulative relative frequency bar chart. 
Check the box \option{Polygon}, to plot the corresponding polygon. 
\end{indication}
\end{enumerate}

\item The number of people treated in the emergency service of a hospital every day of November was
\begin{center}
15 \quad 23 \quad 12 \quad 10 \quad 28 \quad 7 \quad 12 \quad 17 \quad 20 \quad 21 \quad 18 \quad 13 \quad 11 \quad 12 \quad 26 \\
30 \quad 6 \quad 16 \quad 19 \quad 22 \quad 14 \quad 17 \quad 21 \quad 28 \quad 9 \quad 16 \quad 13 \quad 11 \quad 16 \quad 20
\end{center}
Do the following operations: 
\begin{enumerate}
\item Create a data set with the variable \variable{emergencies} and enter the data.

\item Create the box plot. Are there some outlier? In that case, remove the outliers and proceed with the next part.
\begin{indication}
\begin{enumerate}
\item Select the menu \menu{Teaching > charts > Box plot}.
\item In the dialog displayed select the variable \variable{emergencies} in the field \field{Variables} and
click the button \button{Submit}.
\item In the output windows with the box plot identify the outliers.
\item In the data set edition tab, remove the rows with the outliers right-clicking the row header and selecting
\menu{Delete this row}.
\end{enumerate}
\end{indication}

\item Create the frequency table grouping data into 5 classes.
\begin{indication}
\begin{enumerate}
\item Select the menu \menu{Teaching > Frequency distribution > Frequency table}.
\item In the dialog displayed select the variable \variable{emergencies}.
\item In the \option{Classes} tab check the box \option{Grouping intervals}, check the option \option{Number of
intervals}, enter the desired number of intervals in the field \field{Suggested intervals} and
click the button \button{Submit}.
\end{enumerate}
\end{indication}

\item Create the absolute frequency histogram.
\begin{indication}
\begin{enumerate}
\item Select the menu \menu{Teaching > charts > Histogram}.
\item In the dialog displayed select the variable \variable{emergencies} in the field \field{Variable}.
\item In the \option{Classes} tab, check the box \option{Grouping intervals}, check the box
\option{Number of intervals}, enter the desired number of intervals in the field \field{Suggested intervals} and click
the button \button{Submit}.
\end{enumerate}
\end{indication}

\item Create also the relative frequency, cumulative absolute frequency and cumulative relative frequency
histograms, with their respective polygons.
\begin{indication}Follow the steps above checking, in the \option{Histogram options},
the box \option{Relative frequencies} for the relative frequency histogram, the box \option{Cumulative frequencies} for
the cumulative absolute frequency histogram, and both of them for the cumulative relative frequency histogram.
Check the box \option{Polygon} to plot the corresponding polygon. 
\end{indication}
\end{enumerate}

\item The blood type of a sample of 30 persons are:
\begin{center}
A, B, B, A, AB, 0, 0, A, B, B, A, A, A, A, AB,\\
A, A, A, B, 0, B, B, B, A, A, A, 0, A, AB, 0. 
\end{center}
Do the following operations:
\begin{enumerate}
\item Create a data set with the variable \variable{blood.type} and enter the data.

\item Create the frequency table.
\begin{indication}
\begin{enumerate}
\item Select the menu \menu{Teaching > Frequency distribution > Frequency table}.
\item In the dialog displayed, select the variable \variable{blood.type} in the field
\field{Variable to tabulate} and click the button \button{Submit}.
\end{enumerate}
\end{indication}

\item Create the pie chart.
\begin{indication}
\begin{enumerate}
\item Select the menu \menu{Teaching > charts > Pie chart}.
\item In the dialog displayed, select the variable \variable{blood.type} in the field \field{Variables} and click the
button \button{Submit}.
\end{enumerate}
\end{indication}
\end{enumerate}

\item The age and the marital status of a sample of 28 persons are:
\begin{center}
\begin{tabular}{|l|rrrrrrrrr|}
\hline
Marital status & \multicolumn{9}{c|}{Age}\\
\hline
Single    & 31 & 45 & 35 & 65 & 21 & 38 & 62 & 22 & 31 \\
Married     & 72 & 39 & 62 & 59 & 25 & 44 & 54 &    &    \\
Widow(er)      & 80 & 68 & 65 & 40 & 78 & 69 & 75 &    &    \\
Divorced & 31 & 65 & 59 & 58 & 50 &    &    &    &    \\
\hline
\end{tabular}
\end{center}

Do the following operations:
\begin{enumerate}
\item Create a data set with the variables \variable{marital.status} and \variable{age} and enter the data.
\item Create the frequency table of the variable \variable{age} for every marital status.
\begin{indication}
\begin{enumerate}
\item Select the menu \menu{Teaching > Frequency distribution > Frequency table}.
\item In the dialog displayed, select the variable \variable{age} in the field \field{Variable to
tabulate}, check the box \option{Tabulate by groups} and select the variable \variable{marital.status} in the field
\field{Grouping variable(s)}. 
\item In the \option{Classes} tab, check the box \option{Grouping intervals} and click the button \button{Submit}.
\end{enumerate}
\end{indication}

\item Create the box plots of age for every marital status. Are there outliers? Which group have more spread in ages?
\begin{indication}
\begin{enumerate}
\item Select the menu \menu{Teaching > charts > Box plot}.
\item In the dialog displayed, select the variable \variable{age} in the field \field{Variable(s)},
check the box \option{Plot by groups}, select the variable \variable{marital.status} in the field
\field{Grouping variable(s)} and click the button \button{Submit}.
\end{enumerate}
\end{indication}

\item Create the relative frequency histogram of age for every marital status. Compare the histograms.
\begin{indication}
\begin{enumerate}
\item Select the menu \menu{Teaching > charts > Histogram}.
\item In the dialog displayed, select the variable \variable{age} in the field \field{Variable}, check the box
\option{Plot by groups} and select the variable \variable{marital.status} in the field \field{Grouping variable(s)}.
\item In the \option{Classes} tab, check the box \option{Grouping intervals} and click the button \button{Submit}.
\end{enumerate}
\end{indication}
\end{enumerate}

\end{enumerate}


\section{Proposed exercises}
\begin{enumerate}[leftmargin=*]

\item The number of injuries suffered by the members of a soccer team in a league were
\begin{center}
0, 1, 2, 1, 3, 0, 1, 0, 1, 2, 0, 1, 1, 1, 2, 0, 1, 3, 2, 1, 2, 1, 0, 1
\end{center}

Do the following operations:
\begin{enumerate}
\item Construct the frequency table.
\item Create the relative frequency and the cumulative relative frequency bar charts.
\item Create the box plot.
\end{enumerate}

\item The heights (in cm) of 30 students are 
\begin{center}
179, 173, 181, 170, 158, 174, 172, 166, 194, 185,\\
162, 187, 198, 177, 178, 165, 154, 188, 166, 171,\\
175, 182, 167, 169, 172, 186, 172, 176, 168, 187.
\end{center}

Do the following operations:
\begin{enumerate}
\item Create the absolute frequency histogram with classes of width 10 cm from 150 to 200 cm. Are there outliers?
\end{enumerate}

\item The data set \variable{neonates} of the package \variable{rk.Teaching}, contains information about a
sample of 320 newborns that meet the normal gestation time in a hospital during one year.
Do the following operations:
\begin{enumerate}
\item Construct the frequency table of the APGAR score at 1 minute.
If an score less than or equal to 3 indicates that the neonate is depressed, what percentage of neonates is depressed in
the sample?
\item Construct the frequency table of the neonate weight, grouping into classes of width $0.5$ Kg from $2$
to $4.5$ Kg. What intervals contains more neonates?
\item Compare the frequency distribution of the APGAR score at 1 minute for mothers less than 20 years old
and for mothers greater than 20 years old. What group has more depressed neonates?
\item Compare the relative frequency distribution of the neonate weight according to whether the mother smoked or
not during the pregnancy.
If a weight under $2.5$ Kg is considered a low weight, what group has a higher percentage of neonates with low weight?
\item Compute the prevalence of neonates with low weight for smoking and non-smoking mothers during the pregnancy. 
\item Compute the relative risk of low weight of neonate when the mother smokes vs when then mother doesn't smoke
during the pregnancy.
\item Create the bar chart of the APAGAR score at 1 minute.
What is the more common score?
\item Construct the cumulative relative frequency bar chart of the APGAR score at 1 minute.
Below what value is half of the neonates?
\item Compare the relative frequency distribution bar charts of the APGAR score at 1 minute according to whether the mother smoked or
not during the pregnancy. 
What conclusion can be drawn?
\item Construct the histogram of the neonates weights with classes of width $0.5$ Kg from $2$ to $4.5$ Kg.
What class contains more neonates?
\item Compare the relative frequency histograms of the neonates weights, with classes of width $0.5$ Kg from $2$ to
$4.5$, Kg, according to whether the mother smoked or not during the pregnancy. 
What group has neonates with less weight?
\item Compare the relative frequency histograms of the neonates weights, with classes of width $0.5$ Kg from $2$ to
$4.5$, Kg, according to whether the mother smoked or not before the pregnancy. 
What conclusions can be drawn?
\item Construct the box plot of the neonates weights.
What range of weights can be considered normal in the sample?
Are there outliers in the sample?
\item Compare the box plots of the neonates weights according to whether the mother smoked or not during the pregnancy
and whether the mother was less than 20 or greater than 20 years old.
What group has more central spread?
What group has neonates with less weight?
\item Compare the box plots of the APGAR scores at 1 minute and at 5 minutes.
What variable has more central spread?
\end{enumerate}  

\end{enumerate}
