% Author: Alfredo Sánchez Alberca (asalber@ceu.es)

\chapter{Frequency distributions and charts}\label{cha:freqency-distributions}

\section{Solved exercises}
\begin{enumerate}[leftmargin=*]

\item The number of childre in a sample of 25 is
\begin{center}
1, 2, 4, 2, 2, 2, 3, 2, 1, 1, 0, 2, 2, 0, 2, 2, 1, 2, 2, 3, 1, 2, 2, 1, 2.
\end{center}
Do the following operations:
\begin{enumerate}
\item Create a data frame with the variable \variable{children} and enter the data.

\item Create the frequency table.
\begin{indication}
\begin{enumerate}
\item Select the menu \menu{Teaching > Frequency distribution > Frequency tabulation} .
\item In the dialog displayed, select the variable \variable{children} in the field \field{Variable to tabulate} and
click the button \button{Send}.
\end{enumerate}
\end{indication}

\item Create the absolute frequencies bar chart.
\begin{indication}
\begin{enumerate}
\item Select the menu \menu{Teaching > Gráficos > Diagrama de barras}.
\item In the dialog displayed, select the variable \variable{hijos} in the field \field{Variable} and hacer
clic en el botón \button{Send}.
\end{enumerate}
\end{indication}

\item Para la misma tabla de frecuencias anterior, dibujar también el diagrama de barras de las relative frequencies,
el de absolutas acumuladas and el de relativas acumuladas, además de sus correspondientes polígonos.
\begin{indication}Repetir los pasos del apartado anterior activando, en la solapa de \option{Opciones de las barras},
la opción \option{Frecuencias relativas} si se desea el diagrama de barras de relative frequencies, activando la opción
\option{Frecuencias acumuladas} si se desea el diagrama de barras de frecuencias acumuladas and activando la opción
\option{Polígono} para obtener el polígono asociado.
\end{indication}
\end{enumerate}

\item En un hospital se realizó un estudio sobre el número de personas que ingresaron en urgencias cada día del mes de
noviembre. Los datos observados fueron:
\begin{center}
15, 23, 12, 10, 28, 50, 12, 17, 20, 21, 18, 13, 11, 12, 26 \\
30, 6, 16, 19, 22, 14, 17, 21, 28, 9, 16, 13, 11, 16, 20
\end{center}
Se pide:

\begin{enumerate}
\item  Crear un conjunto de datos con la variable \variable{urgencias} e introducir los datos.

\item  Dibujar el diagrama de cajas. ¿Existe algún dato atípico? En el caso de que exista, eliminarlo and proceder con los
siguientes apartados.
\begin{indication}
\begin{enumerate}
\item Select the menu \menu{Teaching > Gráficos > Diagrama de cajas}.
\item In the dialog displayed, select the variable \variable{urgencias} in the field \field{Variables} y
click the button \button{Send}.
\item En la ventana que aparece con el diagrama de cajas identificar el dato atípico.
\item Ir a la ventana de edición de datos and eliminar la fila del dato atípico haciendo clic con el botón derecho del
ratón en la cabecera de la fila and seleccionando \menu{Borrar esta fila}. 
\end{enumerate}
\end{indication}

\item Construir la tabla de frecuencias agrupando en 5 clases.
\begin{indication}
\begin{enumerate}
\item Select the menu \menu{Teaching > Frequency distribution > Frequency tabulation}.
\item In the dialog displayed select the variable \variable{urgencias}.
\item En la solapa de \option{Clases} activar la casilla \option{Agrupar en intervalos}, marcar la opción \option{Número
de intervalos} e introducir el número deseado de intervalos in the field \field{Intervalos sugeridos} and hacer clic sobre el botón
\button{Send}.
\end{enumerate}
\end{indication}

\item  Dibujar el histograma de absolute frequencies correspondiente a la tabla anterior.
\begin{indication}
\begin{enumerate}
\item Select the menu \menu{Teaching > Gráficos > Histograma}.
\item In the dialog displayed select the variable \variable{urgencias} in the field \field{Variable}.
\item En la solapa de \option{Clases} activar la casilla \option{Agrupar en intervalos}, marcar la opción \option{Número
de intervalos} e introducir el número deseado de intervalos in the field \field{Intervalos sugeridos} and hacer clic sobre el botón
\button{Send}.
\end{enumerate}
\end{indication}

\item Para la misma tabla de frecuencias anterior, dibujar también el histograma de las relative frequencies, el de
absolutas acumuladas and el de relativas acumuladas, además de sus correspondientes polígonos.
\begin{indication}Repetir los pasos del apartado anterior activando, en la solapa de \option{Opciones del histograma},
la opción \option{Frecuencias relativas} si se desea el histograma de relative frequencies, activando la opción
\option{Frecuencias acumuladas} si se desea el histograma de frecuencias acumuladas and activando la opción
\option{Polígono} para obtener el polígono asociado.
\end{indication}
\end{enumerate}

\item Los grupos sanguíneos de una muestra de 30 personas son:
\begin{center}
A, B, B, A, AB, 0, 0, A, B, B, A, A, A, A, AB,\\
A, A, A, B, 0, B, B, B, A, A, A, 0, A, AB, 0. 
\end{center}
Se pide:
\begin{enumerate}
\item Crear un conjunto de datos con la variable \variable{grupo.sanguineo} e introducir los datos.

\item Construir la tabla de frecuencias.
\begin{indication}
\begin{enumerate}
\item Select the menu \menu{Teaching > Frequency distribution > Frequency tabulation} .
\item In the dialog displayed, select the variable \variable{grupo.sanguineo} in the field
\field{Variable to tabulate} and click the button \button{Send}.
\end{enumerate}
\end{indication}

\item Dibujar el diagrama de sectores.
\begin{indication}
\begin{enumerate}
\item Select the menu \menu{Teaching > Gráficos > Diagrama de sectores}.
\item In the dialog displayed, select the variable \variable{grupo.sanguineo} in the field
\field{Variables} and hacer clic sobre el botón \button{Send}.
\end{enumerate}
\end{indication}
\end{enumerate}

\item  En un estudio de población se tomó una muestra de 27 personas, and se les preguntó por su edad and estado civil,
obteniendo los siguientes resultados:
\begin{center}
\begin{tabular}{|l|rrrrrrrrr|}
\hline
Estado civil & \multicolumn{9}{c|}{Edad}\\
\hline
Soltero    & 31 & 45 & 35 & 65 & 21 & 38 & 62 & 22 & 31 \\
Casado     & 62 & 39 & 62 & 59 & 21 & 62 &    &    &    \\
Viudo      & 80 & 68 & 65 & 40 & 78 & 69 & 75 &    &    \\
Divorciado & 31 & 65 & 59 & 49 & 65 &    &    &    &    \\
\hline
\end{tabular}
\end{center}

Se pide:
\begin{enumerate}
\item Crear un conjunto de datos con la variables \variable{estado.civil} and \variable{edad} e introducir los datos.
\item Construir la tabla de frecuencias de la variable \variable{edad} para cada categoría de la
variable \variable{estado.civil}.
\begin{indication}
\begin{enumerate}
\item Select the menu \menu{Teaching > Frequency distribution > Frequency tabulation}.
\item In the dialog displayed, select the variable \variable{edad} in the field \field{Variable a
tabular}, activar la casilla \option{Tabular por grupos}, select the variable \variable{estado.civil} in the field
\field{Variable de agrupación} and click the button \button{Send}.
\end{enumerate}
\end{indication}

\item Dibujar los diagramas de cajas de la edad según el estado civil. ¿Existen datos atípicos? ¿En qué grupo hay mayor
dispersión?
\begin{indication}
\begin{enumerate}
\item Select the menu \menu{Teaching > Gráficos > Diagrama de cajas}.
\item In the dialog displayed, select the variable \variable{edad} in the field \field{Variables},
activar la casilla \option{Dibujar por grupos}, select the variable \variable{estado.civil} in the field
\field{Variable de agrupación} and click the button \button{Send}.
\end{enumerate}
\end{indication}
\end{enumerate}

\end{enumerate}


\section{Ejercicios propuestos}
\begin{enumerate}[leftmargin=*]

\item  El número de lesiones padecidas durante una temporada por cada jugador de un equipo de fútbol fue el siguiente:
\begin{center}
0, 1, 2, 1, 3, 0, 1, 0, 1, 2, 0, 1, 1, 1, 2, 0, 1, 3, 2, 1, 2, 1, 0, 1
\end{center}

Se pide:
\begin{enumerate}
\item Construir la tabla de frecuencias.
\item Dibujar el diagrama de barras de las relative frequencies and de relative frequencies acumuladas.
\item Dibujar el diagrama de sectores.
\end{enumerate}

\item Para realizar un estudio sobre la estatura de los estudiantes universitarios, seleccionamos, mediante un proceso
de muestreo aleatorio, una muestra de 30 estudiantes, obteniendo los siguientes resultados (medidos en centímetros):
\begin{center}
179, 173, 181, 170, 158, 174, 172, 166, 194, 185,\\
162, 187, 198, 177, 178, 165, 154, 188, 166, 171,\\
175, 182, 167, 169, 172, 186, 172, 176, 168, 187.
\end{center}

Se pide:
\begin{enumerate}
\item Dibujar el histograma de las absolute frequencies agrupando desde 150 a 200 en clases de amplitud 10.
\item Dibujar el diagrama de cajas. ¿Existe algún dato atípico?.
\end{enumerate}

\item El conjunto de datos \variable{neonatos} del paquete \variable{rk.Teaching}, contiene información sobre una
muestra de 320 recién nacidos en un hospital durante un año que cumplieron el tiempo normal de gestación. 
Se pide:
\begin{enumerate}
\item Construir la tabla de frecuencias de la puntuación Apgar al minuto de nacer. 
Si se considera que una puntuación Apgar de 3 o menos indica que el neonato está deprimido, ¿qué porcentaje de niños está deprimido en la muestra?
\item Comparar las distribuciones de frecuencias de las puntuaciones Apgar al minuto de nacer según si la madre es mayor
o menor de 20 años.
¿En qué grupo hay más neonatos deprimidos?
\item Construir la tabla de frecuencias para el peso de los neonatos, agrupando en clases de amplitud $0.5$ desde el
$2$ hasta el $4.5$. ¿En qué intervalo de peso hay más niños?
\item Comparar la distribución de relative frequencies del peso de los neonatos según si la madre fuma o no. Si se
considera como peso bajo un peso menor de $2.5$ kg, ¿En qué grupo hay un mayor porcentaje de niños con peso bajo?
\item Si en los recién nacidos se considera como peso bajo un peso menor de $2.5$ kg, calcular la prevalencia del bajo
peso de recién nacidos en el grupo de madres fumadoras and en el de no fumadoras. 
\item Calcular el riesgo relativo de que un recién nacido tenga bajo peso cuando la madre fuma, frente a cuando la madre
no fuma. 
\item Construir el diagrama de barras de la puntuación Apgar al minuto. ¿Qué puntuación Apgar es la más frecuente? 
\item Construir el diagrama de relative frequencies acumuladas de la puntuación Apgar al minuto. ¿Por debajo de que puntuación estarán la mitad de los niños?
\item Comparar mediante diagramas de barras de relative frequencies las distribuciones de las puntuaciones Apgar al
minuto según si la madre ha fumado o no durante el embarazo. ¿Qué se puede concluir?
\item Construir el histograma de pesos, agrupando en clases de amplitud $0.5$ desde el $2$ hasta el $4.5$. ¿En qué
intervalo de peso hay más niños?
\item Comparar la distribución de relative frequencies del peso de los neonatos según si la madre fuma o no. ¿En qué
grupo se aprecia menor peso de los niños de la muestra?
\item Comparar la distribución de relative frequencies del peso de los neonatos según si la madre fumaba o no antes del
embarazo. ¿Qué se puede concluir?
\item Construir el diagrama de caja and bigotes del peso. ¿Entre qué valores se considera que el peso de un neonato es
normal? ¿Existen datos atípicos?
\item Comparar el diagrama de cajas and bigotes del peso, según si la madre fumó o no durante el embarazo and si era mayor o
no de 20 años. ¿En qué grupo el peso tiene más dispersión central? ¿En qué grupo pesan menos los niños de la
muestra?
\item Comparar el diagrama de cajas de la puntuación Apgar al minuto and a los cinco minutos. ¿En qué variable hay más
dispersión central?
\end{enumerate}  

\end{enumerate}
