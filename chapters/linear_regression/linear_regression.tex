% Author: Alfredo Sánchez Alberca (asalber@ceu.es)

\chapter{Linear regression}\label{cha:linear-regression}

\section{Solved exercises}
\begin{enumerate}[leftmargin=*]
\item The values of two variables $X$ and $Y$ measured in a sample of 10 individuals are:
\[
\begin{array}{lrrrrrrrrrr}
\hline
X & 0 & 1 & 2 & 3 & 4 & 5 & 6 & 7 & 8 & 9 \\
Y & 2 & 5 & 8 & 11 & 14 & 17 & 20 & 23 & 26 & 29\\
\hline
\end{array}
\]

Do the following operations:

\begin{enumerate}
\item Create a data set with the variables \variable{X} and \variable{Y} amd enter the data.
\item Construct the scatter plot of \variable{X} and \variable{Y}.
\begin{indication}
\begin{enumerate}
\item Select the menu \menu{Teaching > Charts > Scatter plot}.
\item In the dialog displayed, select the variable \variable{Y} in the field \field{Variable Y}, and the variable
\variable{X} in the field \field{Variable X}, and click the button \button{Submit}.
\end{enumerate}
\end{indication}

According to the point cloud, what type of regression model explains better the relation between \variable{X} and
\variable{Y}?

\item Compute the regression line of $Y$ on $X$.
\begin{indication}
\begin{enumerate}
\item Select the menu \menu{Teaching > Regression > Linear regression}.
\item In the dialog displayed, insert the variable \variable{Y} in the field \field{Dependent variable} and the variable
\variable{X} in the field \field{Independent variable}, and click the button \button{Submit}.
\end{enumerate}
\end{indication}

\item Plot the regression line on the scatter plot.
\begin{indication}
\begin{enumerate}
\item Select the menu \menu{Teaching > Charts > Scatter plot}.
\item In the dialog displayed, insert the variable \variable{Y} in the field \field{Variable Y} and the
variable \variable{X} in the field \field{Variable X}.
\item In the \menu{Fitted line} tab, check the box \option{Linear} and click the button
\button{Submit}.
\end{enumerate}
\end{indication}

\item Compute la regression line of $X$ on $Y$ and plot it on the scatter plot.
\begin{indication}
Repeat the steps of the previous part but inserting the variable \variable{X} in the field \field{Dependent variable}
and the variable \variable{Y} in the field \field{Independent variable}. 
\end{indication}

\item How are the residuals?
Comment the results.
\end{enumerate}


\item  En una licenciatura se quiere estudiar la relación entre el número medio de horas de estudio diarias and el número
de asignaturas suspensas. Para ello se obtuvo la siguiente muestra:
\[
\begin{array}{cccccccc}
\text{Horas} & \text{Suspensos} &  & \text{Horas} & \text{Suspensos} & & \text{Horas} & \text{Suspensos}  \\
\cline{1-2}\cline{4-5}\cline{7-8}
3.5 & 1 & & 2.2 & 2 & & 1.3 & 4 \\
0.6 & 5 & & 3.3 & 0 & & 3.1 & 0 \\
2.8 & 1 & & 1.7 & 3 & & 2.3 & 2 \\
2.5 & 3 & & 1.1 & 3 & & 3.2 & 2 \\
2.6 & 1 & & 2.0 & 3 & & 0.9 & 4 \\
3.9 & 0 & & 3.5 & 0 & & 1.7 & 2 \\
1.5 & 3 & & 2.1 & 2 & & 0.2 & 5 \\
0.7 & 3 & & 1.8 & 2 & & 2.9 & 1 \\
3.6 & 1 & & 1.1 & 4 & & 1.0 & 3 \\
3.7 & 1 & & 0.7 & 4 & & 2.3 & 2 \\
\end{array}
\]

Do the following operations:
\begin{enumerate}
\item  Create a data set con las variables \variable{horas.estudio} and \variable{suspensos} e introducir estos
datos.

\item Construir la frequency table bidimensional de las variables \variable{horas.estudio} and \variable{suspensos}. 
\begin{indication}
\begin{enumerate}
\item Select the menu \menu{Teaching > Frequency distribution > Tabla de frecuencias bidimensional}.
\item In the dialog displayed, select the variable \variable{horas.estudio} in the field \field{Variable
a tabular en filas}, la variable \variable{suspensos} in the field \field{Variable a tabular en columnas}, and hacer clic
sobre el botón \button{Submit}. 
\end{enumerate}
\end{indication}

\item  Compute la regression line of \variable{suspensos} sobre \variable{horas.estudio} and dibujarla.
\begin{indication}
Para calcular la recta de regresión:
\begin{enumerate}
\item Select the menu \menu{Teaching > Regression > Linear regression}.
\item In the dialog displayed, select the variable \variable{suspensos} in the field \field{Variable
dependiente} and la variable \variable{horas.estudio} in the field \field{Independent variable}, seleccionar
\option{Guardar el modelo}, introducir un nombre para el modelo and click the button \button{Submit}.
\end{enumerate}
Para dibujar la recta de regresión:
\begin{enumerate}
\item Select the menu \menu{Teaching > Charts > Scatter plot}.
\item In the dialog displayed, select the variable \variable{suspensos} in the field \field{Variable Y} y
la variable \variable{horas.estudio} in the field \field{Variable X}.
\item En la solapa \menu{Fitted line}, seleccionar \option{Lineal} and click the button
\button{Submit}.
\end{enumerate}
\end{indication}

\item Indicar el coeficiente de regresión de \variable{suspensos} sobre \variable{horas.estudio}. 
¿Cómo lo interpretarías?
\begin{indication}
El coeficiente de regresión es la pendiente de la recta de regresión.
\end{indication}

\item La relación lineal entre estas dos variables, ¿es mejor o peor que la del ejercicio anterior? 
Comentar los resultados a partir las gráficas de las rectas de regresión and sus residuos.

\item Compute los coeficientes de correlación and de determinación lineal. 
¿Es un buen modelo la recta de regresión?
¿Qué porcentaje de la variabilidad del número de suspensos está explicada por el modelo?
\begin{indication}
El coeficiente de determinación aparece en la ventana de resultados como \result{R$^2$}, and el
coeficiente de correlación es su raíz cuadrada.
\end{indication}

\item Utilizar la recta de regresión para predecir el número de suspensos correspondiente a 3 horas de estudio diarias.
¿Es fiable esta predicción? 
\begin{indication}
\begin{enumerate}
\item Select the menu \menu{Teaching > Regression > Predicciones}.
\item In the dialog displayed seleccionar como modelo de regresión la recta calculada en el segundo
apartado, introducir los valores para los que se desea la predicción in the field \field{Predicciones para} and hacer clic
sobre el botón \button{Submit}.
\end{enumerate}
\end{indication}

\item Según el modelo lineal, ¿cuántas horas diarias tendrá que estudiar como mínimo un alumno si quiere aprobarlo
todo?
\begin{indication}
Seguir los mismos pasos de los apartados anteriores, pero escogiendo como variable dependiente \variable{horas.estudio},
y como independiente \variable{suspensos}, and haciendo la predicción para 0 suspensos.
\end{indication}
\end{enumerate}


\item Después de tomar un litro de vino se ha medido la concentración de alcohol en la sangre en distintos instantes,
obteniendo:
\[
\begin{array}{lrrrrrrr}
\hline 
\mbox{Tiempo después (minutos)} & 30 & 60 & 90 & 120 & 150 & 180 & 210\\ 
\mbox{Concentración (gramos/litro)} & 1.6 & 1.7 & 1.5 & 1.1 & 0.7 & 0.2 & 2.1\\
\hline
\end{array}
\]

Do the following operations:
\begin{enumerate}
\item Crear las variables \variable{tiempo} and \variable{alcohol} e introducir estos datos.
\item Compute el coeficiente de correlación lineal entre el alcohol and el tiempo e interpretarlo. ¿Es bueno el modelo
lineal? 
\begin{indication}
\begin{enumerate}
\item Select the menu \menu{Teaching > Regression > Linear regression}.
\item In the dialog displayed, select the variable \variable{alcohol} in the field \field{Variable
dependiente} and la variable \variable{tiempo} in the field \field{Independent variable}, and click the button
\button{Submit}.
\end{enumerate}
\end{indication}

\item Dibujar la regression line ofl alcohol sobre el tiempo. 
¿Existe algún individuo con un residuo demasiado grande? 
Si es así, eliminar dicho individuo de la muestra and volver a calcular el coeficiente de correlación. 
¿Ha mejorado el modelo?
\begin{indication}
\begin{enumerate}
\item Select the menu \menu{Teaching > Charts > Scatter plot}.
\item In the dialog displayed, select the variable \variable{alcohol} in the field \field{Variable Y} y
la variable \variable{tiempo} in the field \field{Variable X}.
\item En la solapa \menu{Fitted line}, seleccionar \option{Lineal} and click the button \button{Submit}.
\end{enumerate}
Se observa que hay un residuo atípico para el punto que corresponde al los 210 minutos. 
Para eliminarlo:
En la ventana de edición del conjunto de datos hacer clic con el botón derecho del ratón sobre la fila correspondiente
al dato con el residuo atípico and seleccionar \option{Borrar esta fila}.
\end{indication}

\item Si la concentración máxima de alcohol en la sangre que permite la ley para poder conducir es $0.3$ g/l, ¿cuánto
tiempo habrá que esperar después de tomarse un litro de vino para poder conducir sin infringir la ley? 
¿Es fiable esta predicción?
\begin{indication}
Para construir la recta de regresión:
\begin{enumerate}
\item Select the menu \menu{Teaching > Regression > Linear regression}.
\item In the dialog displayed, select the variable \variable{tiempo} in the field \field{Variable
dependiente} and la variable \variable{alcohol} in the field \field{Independent variable}.
\item Seleccionar \option{Guardar el modelo}, introducir un nombre para el modelo and click the button \button{Submit}.
\end{enumerate}
Para hacer la predicción:
\begin{enumerate}
\item Select the menu \menu{Teaching > Regression > Predicciones}.
\item In the dialog displayed seleccionar como modelo de regresión la recta calculada e introducir los
valores para los que se desea la predicción in the field \field{Predicciones para} and click the button
\button{Submit}.
\end{enumerate}
\end{indication}
\end{enumerate}


\item El conjunto de datos \variable{edad.estatura} del paquete \variable{rk.Teaching} contine la edad and la estatura
de 30 personas. 
Do the following operations:
\begin{enumerate}
\item Cargar datos del conjunto de datos \variable{edad.estatura} desde el paquete \variable{rk.Teaching}.

\item Compute la regression line of la estatura sobre la edad. ¿Es un buen modelo la recta de regresión?
\begin{indication}
\begin{enumerate}
\item Select the menu \menu{Teaching > Regression > Linear regression}.
\item In the dialog displayed, select the variable \variable{estatura} in the field \field{Variable
dependiente} and la variable \variable{edad} in the field \field{Independent variable}, and click the button
\button{Submit}.
\end{enumerate}
\end{indication}

\item Dibujar el diagrama de dispersión de la estatura sobre la edad. 
¿Alrededor de qué edad se observa un cambio en la tendencia? 
\begin{indication}
\begin{enumerate}
\item Select the menu \menu{Teaching > Charts > Scatter plot}.
\item In the dialog displayed, select the variable \variable{estatura} in the field \field{Variable Y},
la variable \variable{edad} in the field \field{Variable X}, and click the button \button{Submit}.
\end{enumerate}
\end{indication}

\item Recodificar la variable edad en dos grupos para mayores and menores de 20 años.
\begin{indication}
\begin{enumerate}
\item Select the menu \menu{Teaching >  Datos > Recodificar variable}.
\item In the dialog displayed seleccionar in the field \vampo{Variable a recodificar} la variable
\variable{edad}.
\item En el campo \field{Reglas de recodificación} introducir
\begin{quote}
\lstinline{lo:20 = "menores"}\\
\lstinline{20:hi = "mayores"}
\end{quote}
\item En el cuadro \field{Guardar nueva variable} click the button \button{Cambiar}.
\item In the dialog displayed seleccionar como objeto padre la el conjunto de datos \variable{edad\_estatura} and click the button \button{Aceptar}.
\item Introducir el nombre de la nueva variable \variable{grupo.edad} and click the button \button{Submit}.
\end{enumerate}
\end{indication}

\item Compute la regression line of la estatura sobre la edad para cada grupo de edad. 
¿En qué grupo explica mejor la recta de regresión la relación entre la estatura and la edad? 
Justificar la respuesta.
\begin{indication}
\begin{enumerate}
\item Select the menu \menu{Teaching > Regression > Linear regression}.
\item In the dialog displayed, select the variable \variable{estatura} in the field \field{Variable
dependiente} and la variable \variable{edad} como \field{Independent variable}.
\item Seleccionar la opición \option{Ajuste por grupos}, introducir la variable \variable{grupo.edad} in the field
\field{Grouping variable(s)}, and hacer clic en el \button{Submit}.
\end{enumerate}
\end{indication}

\item Dibujar las rectas de regresión anteriores.
\begin{indication}
\begin{enumerate}
\item Select the menu \menu{Teaching > Charts > Scatter plot}.
\item In the dialog displayed, select the variable \variable{estatura} in the field \field{Variable Y} y
la variable \variable{edad} in the field \field{Variable X}.
\item Seleccionar la opción \option{Plot by groups} e introducir la variable \variable{grupo.edad} in the field
\field{Grouping variable(s)}.
\item En la solapa \menu{Fitted line}, seleccionar \option{Lineal} and click the button
\button{Submit}.
\end{enumerate}
\end{indication}

\item ¿Qué estatura se espera que tenga una persona de 14 años? ¿Y una de 38?
\begin{indication}
Para predecir la estatura de la persona de 14 años:
\begin{enumerate}
\item Select the menu \menu{Teaching > Regression > Predicciones}.
\item In the dialog displayed seleccionar como modelo de regresión la recta calculada para los menores e
introducir 14 in the field \field{Predicciones para} and click the button
\button{Submit}.
\end{enumerate}
para predecir la estatura de la persona de 38 años, repetir lo mismo pero seleccionando la recta de regresión para los
mayores e introducidento 38 in the field \field{Predicciones para}.
\end{indication}
\end{enumerate}

\opt{largo}{
\item La siguiente tabla recoge la información de las calificaciones obtenidas por un grupo de alumnos en dos
asignaturas $X$ e $Y$.
\begin{center}
\begin{tabular}{lcccccccccccc}
Alumno & 1 & 2 & 3 & 4 & 5 & 6 & 7 & 8 & 9 & 10 & 11 & 12\\
\hline
$X$ & NT & AP & SS & SS & AP & AP & SS & NT & SB & SS & AP & AP\\
$Y$ & SB & SS & AP & SS & AP & NT & SS & NT & NT & AP & AP & NT
\end{tabular}
\end{center}
Do the following operations:
\begin{enumerate}
\item Create a data set con las variables \varaible{X} e \variable{Y} and enter the data.

\item ¿Existe relación entre las calificaciones de $X$ e $Y$? Justificar la respuesta.
\begin{indication}
\begin{enumerate}
\item Select the menu \menu{Teaching > Regression > Correlación}.
\item In the dialog displayed select the variables \variable{X} e \variable{Y} in the field
\field{Variables}.
\item En la solapa \menu{Opciones de correlación} seleccionar el método de \option{Ro de Spearman} and hacer clic sobre
el botón \button{Submit}.
\end{enumerate}
\end{indication}
\end{enumerate}
}
\end{enumerate}


\section{Proposed exercises}
\begin{enumerate}[leftmargin=*]
\item  Se determina la pérdida de actividad que experimenta un medicamento desde el momento de su fabricación a lo
largo del tiempo, obteniéndose el siguiente resultado:
\begin{center}
\begin{tabular}{|l|c|c|c|c|c|}
\hline 
Tiempo (en años) & 1 & 2 & 3 & 4 & 5 \\ 
\hline 
Actividad restante (\%) & 96 & 84 & 70 & 58 & 52 \\ 
\hline
\end{tabular}
\end{center}
Se desea calcular:
\begin{enumerate}
\item  La relación fundamental (recta de regresión) entre actividad restante and tiempo transcurrido.
\item ¿En qué porcentaje disminuye la actividad cada año que pasa?
\item ¿Cuándo tiempo debe pasar para que el fármaco tenga una actividad del 80\%? ¿Cuándo será nula la actividad?
¿Son igualmente fiables estas predicciones?
\end{enumerate}

\item Al realizar un estudio sobre la dosificación de un cierto medicamento, se trataron 6 pacientes con dosis diarias
de 2 mg, 7 pacientes con 3 mg and otros 7 pacientes con 4 mg. De los pacientes tratados con 2 mg, 2 curaron al cabo de 5
días, and 4 al cabo de 6 días. De los pacientes tratados con 3 mg diarios, 2 curaron al cabo de 3 días, 4 al cabo de 5
días and 1 al cabo de 6 días. Y de los pacientes tratados con 4 mg diarios, 5 curaron al cabo de 3 días and 2 al cabo de 4
días. Do the following operations: 
\begin{enumerate}
\item Compute la regression line ofl tiempo de curación con respecto a la dosis suministrada.
\item Compute el coeficiente de regresión del tiempo de curación con respecto a la dosis e interpretarlo.
\item Compute el coeficiente de correlación lineal e interpretarlo.
\item Determinar el tiempo esperado de curación para una dosis de 5 mg diarios. ¿Es fiable esta predicción?
\item ¿Qué dosis debe aplicarse si queremos que el paciente tarde 4 días en curarse? ¿Es fiable la predicción?
\end{enumerate}

\item El fichero \variable{estaturas.pesos.alumnos} del paquete \variable{rk.Teaching}, contiene la estatura, el peso y
el sexo de una muestra de alumnos universitarios.
Do the following operations:
\begin{enumerate}
\item Cargar el conjunto de datos \variable{estaturas.pesos.alumnos} desde el paquete \variable{rk.Teaching}.
\item Compute la regression line ofl peso sobre la estatura and dibujarla.
\item Compute las rectas de regresión del peso sobre la estatura para cada sexo and dibujarlas.
\item Compute los coeficientes de determinación de ambas rectas. ¿Qué recta es mejor modelo? Justificar la respuesta.
\item ¿Qué peso tendrá un hombre que mida 170 cm? ¿Y una mujer de la misma estatura?
\end{enumerate}

\item El conjunto de datos \variable{neonatos} del paquete \variable{rk.Teaching}, contiene información sobre una
muestra de 320 recién nacidos en un hospital durante un año que cumplieron el tiempo normal de gestación. 
Do the following operations:
\begin{enumerate}
\item Construir la frequency table bidimensional del Agpar al minuto de nacer frente a si la madre ha fumado o no
durante el embarazo. ¿Qué conclusiones se pueden sacar?
\item Construir la frequency table bidimensional del peso de los recién nacidos frente a la edad de la madre. ¿Qué
conclusiones se pueden sacar?
\item Construir la regression line ofl peso de los recién nacidos sobre el número de cigarros fumados al día por las
madres. ¿Existe una relación lineal fuerte entre el peso and el número de cigarros?
\item Dibujar la recta de regresión calculada en el apartado anterior. ¿Por qué la recta no se ajusta bien a la nube de
puntos?
\item Compute and dibujar la regression line ofl peso de los recién nacidos sobre el número de cigarros fumados al día
por las madres en el grupo de las madres que si fumaron durante el embarazo. ¿Es este modelo mejor o pero que la recta
de los apartados anteriores? 

Según este modelo, ¿cuánto disminuirá el peso del recién nacido por cada cigarro más diario
que fume la madre? 
\item Según el modelo anterior, ¿qué peso tendrá un recién nacido de una madre que ha fumado 5 cigarros diarios durante
el embarazo? ¿Y si la madre ha fumado 30 cigarros diarios durante el embarazo? ¿Son fiables estas predicciones?
\item ¿Existe la misma relación lineal entre el peso de los recién nacidos and el número de cigarros fumados al día por
las madres que fumaron durante el embarazo en el grupo de las madres menores de 20 and en el grupo de las madres mayores de
20? ¿Qué se puede concluir?
\end{enumerate}
\end{enumerate}