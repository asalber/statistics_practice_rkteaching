% Author: Alfredo Sánchez Alberca (asalber@ceu.es)

\chapter{Confidence intervals for one population}\label{cha:confidence-intervals-1}

\section{Solved exercises}
\begin{enumerate}[leftmargin=*]
\item  The active ingredient concentration of a random sample of 10 drug containers drawn from a batch are (in
mg/mm$^{3}$)
\[
17.6\quad 19.2\quad 21.3\quad 15.1\quad 17.6\quad 18.9\quad 16.2\quad 18.3\quad 19.0\quad 16.4
\]

Do the following operations:
\begin{enumerate}
\item Create a data set with the variable \variable{concentration} and enter the data of the sample.

\item  Compute the confidence interval for the mean of the active ingredient concentration with a 95\% confidence level
(significance level $\alpha =0.05$).
\begin{indication}
\begin{enumerate}
\item Select the menu \menu{Teaching > Parametric tests > Means > t-test for the mean of one population}.
\item In the dialog displayed insert the variable \variable{concentration} in the field \field{Mean of}
and click the button \button{Submit}.
\end{enumerate}
\end{indication}

\item Compute the confidence interval for the mean of the active ingredient concentration 99\% confidence level
(significance level $\alpha=0.01$).
\begin{indication}
\begin{enumerate}
\item Select the menu \menu{Teaching > Parametric tests > Means > t-test for the mean of one population}.
\item In the dialog displayed insert the variable \variable{concentration} in the field \field{Mean of}·
\item In the \mtab{Test options} tab enter 0.99 in the \field{Confidence level} and click the button \button{Submit}.
\end{enumerate}
\end{indication}

\item If we define the precision of the interval as the inverse of the its width, how changes the precision of an
interval when we increase the confidence level? Why?

\item What sample size is required to get an estimate of the mean of the active ingredient concentration with an error 
$\pm 0.5$ mg/mm$^3$ and a 95\% confidence level?
\begin{indication}
\begin{enumerate}
\item Select the menu \menu{Teaching > Descriptive statistics > Statistics}.
\item In the dialog displayed insert the variable \variable{concentration} in the field \field{Mean of}.
\item En the \mtab{Basic statistics} check the box \field{Corrected standard deviation} and click the button
\button{Submit}.
\item Select the menu \menu{Teaching > Parametric tests > Means > Sample size to estimate one mean}.
\item In the dialog displayed insert the corrected standard deviation in the field \field{Standard deviation}, enter
$0.05$ in the field \field{Significance level}, enter $0.5$ in the field \field{Error} and click the button \button{Submit}.
\end{enumerate}
\end{indication}

\item If the active ingredient concentration must be at least 16 mg/mm$^3$ in order to be effective, can we validate the
batch?
Justify the answer. 
\end{enumerate}

\item A diary company receive milk from two farms $X$ and $Y$.
To analyze the quality of milk, the milk fat have been measure for two samples of milk, one from each farm.
The results are in the table below. 
\[
\begin{array}{ll|ll}
\multicolumn{2}{c|}{X} & \multicolumn{2}{c}{Y} \\
\hline
0.34 & 0.34 & 0.28 & 0.29 \\
0.32 & 0.35 & 0.30 & 0.32 \\
0.33 & 0.33 & 0.32 & 0.31 \\
0.32 & 0.32 & 0.29 & 0.29 \\
0.33 & 0.30 & 0.31 & 0.32 \\
0.31 & 0.32 & 0.29 & 0.31 \\
 &  & 0.33 & 0.32 \\
 &  & 0.32 & 0.33 \\
\end{array}
\]

\begin{enumerate}
\item Create a data set with the variables \variable{fat} and \variable{farm} and enter the data of the sample.

\item Compute the 95\% confidence interval for the mean of fat regardless the farm. 
\begin{indication}
\begin{enumerate}
\item Select the menu \menu{Teaching > Parametric tests > Means > t-test for the mean of one population}.
\item In the dialog displayed insert the variable \variable{fat} in the field \field{Mean of} and click the button
\button{Submit}.
\end{enumerate}
\end{indication}

\item Compute the 95\% confidence intervals for the mean of fat for every farm. 
\begin{indication}
\begin{enumerate}
\item Select the menu \menu{Teaching > Parametric tests > Means > t-test for the mean of one population}.
\item In the dialog displayed select the variable \variable{fat} in the field \field{Mean of}.
\item Check the box \option{Means by groups}, insert the variable \variable{farm} in the field \field{Grouping
variables(s)} and click the button \button{Submit}.
\end{enumerate}
\end{indication}

\item Plot the 95\% confidence intervals for the mean of fat for every farm. 
\begin{indication}
\begin{enumerate}
\item Select the menu \menu{Teaching > Charts > Means plot}.
\item In the dialog displayed select the variable \variable{fat} in the field \field{Mean(s) of}.
\item Check the box \option{Plot by groups}, insert the variable \variable{farm} in the field \field{Grouping
variable(s)} and click the button \button{Submit}.
\end{enumerate}
\end{indication}

\item Is there a significant difference between the milk fat means of the farms?
Justify the answer. 
\end{enumerate}


\item In a survey performed by a university about the use of the library, a random sample of 34 students has been asked
whether they go to the library at least once a week.
The answers are shown below. 
\begin{center}
\begin{tabular}{lllllllllllllllll}
no & yes & no & no & no & yes & no & yes & yes & yes & yes & no & yes & no & yes & no & no \\
no & yes & yes & yes & no & no & yes & no & no & yes & yes & no & no & yes & no & yes & no \\
\end{tabular}
\end{center}

\begin{enumerate}
\item Create a data set with the variable \variable{answer} and enter the data of the sample.
\item Compute the confidence interval for the proportion of students that uses the library at least once a week with a
significance level $0.01$, and interpret it. 
\begin{indication}
\begin{enumerate}
\item Select the menu \menu{Teaching> Parametric tests > Proportions > Test for one proportion}.
\item In the dialog displayed insert the variable \variable{answer} in the field \field{Variable} and enter \texttt{yes}
in the field \field{Proportion of}.
\item In the \mtab{Test options} tab enter $0.99$ in the field \field{Confidence level} and click the button \button{Submit}.
\end{enumerate}
\end{indication}

\item How is the precision of the confidence interval?

\item What sample size is required to get an estimate of the proportion of students that uses the library at least once
a week with and error $\pm 1\%$ and a 95\% confidence level?
\begin{indication}
\begin{enumerate}
\item Select the menu \menu{Teaching > Parametric tests > Proportions > Sample size to estimate one proportion}.
\item In the dialog displayed enter the sample proportion in the field \field{p}, enter $0.05$ in the field
\field{Significance level}, enter $0.01$ in the field \field{Error} and click the button \button{Submit}.
\end{enumerate}
\end{indication}
\end{enumerate}


\item The Ministry of Health wants to compute a confidence interval for the proportion of people over 65 that
with respiratory problems that have been vaccinated. 
In a random sample of 200 persons over 65 with respiratory problems, 154 were vaccinated.  
\begin{enumerate}
\item Compute el 95\% confidence interval for the proportion of people over 65 with respiratory problems vaccinated.
\begin{indication}
\begin{enumerate}
\item Select the menu \menu{Teaching > Parametric tests > Proportions > Test for one proportion}.
\item In the dialog displayed check the box \option{Manual entry of frequencies}, enter 154 in the field
\field{Sample frequency}, enter 200 in the field \field{Sample size} and click the button \button{Submit}.
\end{enumerate}
\end{indication}

\item If Ministry of Health goal si to achieve at least a 70\% of people over 65 with respiratory problems
vaccinated, can we say that the Ministry has achieved the goal?
Justify the answer.
\end{enumerate}
\end{enumerate}


\section{Proposed exercises}
\begin{enumerate}[leftmargin=*] 
\item  The level of cholesterol (in mg/dl) of a random sample of 8 persons of a population is
\begin{center}
196\quad 212\quad 188\quad 206\quad 203\quad 210\quad 201\quad 198
\end{center}

Compute the confidence intervals for the mean with significance levels $0.1$, $0.05$ and $0.01$. 
Can we conclude that the mean of the level of cholesterol of the population is under 210 mg/dl?

\item To treat a neurological syndrome there are two techniques $A$ and $B$.
In a study a random sample of 60 persons were drawn.
Technique $A$ was applied to 25 of them and technique $B$ to the others 35.
18 of the persons treated with $A$ were cured, while 21 of the persons treated with $B$ were cured. 
Compute the confidence interval for the proportion of persons that were cured with every technique. 
Which interval is more precise?

% \item A las siguientes elecciones locales en una ciudad se presentan tres partidos: A, B and C. Con el objetivo de hacer
% una estimación sobre la proporción de voto que cada uno de ellos obtendrá, se realiza una encuesta en la que responden
% 300 personas, de las cuales 60 piensan votar a A, 80 a B, 90 a C, 15 en blanco and 55 abstenciones. Compute un intervalo
% de confianza para la proporción de votos, sobre el total del censo, de cada uno de los partidos que se presentan.

\item The data set \variable{neonates} of the package \variable{rk.Teaching}, contains information about a
sample of 320 newborns that meet the normal gestation time in a hospital during one year.
Do the following operations:
\begin{enumerate}
\item Compute the 99\% confidence interval for the mean of the neonates weight.
\item Compute the confidence intervals for the APGAR score at 1 minute and for the APGAR score at 5 minutes and compare
them. 
Is there a significant difference between the means of both scores?
\item Compute the confidence intervals for the percentage of neonates with weight less than or equal to 2.5 Kg for
smoker and non-smoker mothers and compare them.
\end{enumerate}
\end{enumerate}
