% Author: Alfredo Sánchez Alberca (asalber@ceu.es)

\chapter{Hypothesis tests}\label{cha:hypothesis-tests}

\section{Solved exercises}
\begin{enumerate}[leftmargin=*]
% \item Para averiguar si en una determinada población existen menos hombres que mujeres se plantea un contraste de
% hipótesis sobre la proporción de hombres que hay en la población: $H_0:\ p=0.5$ frente a $H_1:\ p<0.5$ y para ello se
% toma una muestra aleatoria de 10 personas. 
% Se pide:
% \begin{enumerate}
% \item Suponiendo cierta la hipótesis nula, ¿qué distribución sigue la variable que mide el número de hombres en la
% muestra de tamaño 10?
% \item Suponiendo cierta la hipótesis nula, ¿cuál es la probabilidad de que en la muestra se obtengan 0 hombres?
% ¿Se aceptaría la hipótesis nula en tal caso? 
% Justificar la respuesta.
% \begin{indication}
% \begin{enumerate}
% \item Select the menu \menu{Teaching > Distribuciones > Discretas > Binomial > Probabili\-dades acumuladas}.
% \item In the dialog displayed, introducir 0 en el campo \field{Valor(es) de la variable}, 10 en el campo
% \field{Nº de repeticiones}, $0.5$ en el campo \field{Probabilidad de éxito}, marcar la opción \opcion{Cola izquierda} y
% hacer click en el botón \boton{Aceptar}.
% \end{enumerate}
% \end{indication}
% 
% \item Suponiendo cierta la hipótesis nula, si se decide rechazarla cuando en la muestra haya 2 o menos hombres, ¿cuál es
% el riesgo de equivocarse?
% \begin{indication}
% \begin{enumerate}
% \item Select the menu \menu{Teaching > Distribuciones > Discretas > Binomial > Probabili\-dades acumuladas}.
% \item In the dialog displayed, introducir 2 en el campo \field{Valor(es) de la variable}, 10 en el campo
% \field{Nº de repeticiones}, $0.5$ en el campo \field{Probabilidad de éxito}, marcar la opción \opcion{Cola izquierda} y
% hacer click en el botón \boton{Aceptar}.
% \end{enumerate}
% \end{indication}
% 
% \item Si el máximo riesgo de error $\alpha$ que se tolera es $0.05$, ¿qué número de hombres en la muestra formarían la
% región de rechazo de la hipótesis nula?
% \begin{indication}
% \begin{enumerate}
% \item Select the menu \menu{Teaching > Distribuciones > Discretas > Binomial > Probabili\-dades acumuladas}.
% \item In the dialog displayed, introducir 1 en el campo \field{Valor(es) de la variable}, 10 en el campo
% \field{Nº de repeticiones}, $0.5$ en el campo \field{Probabilidad de éxito}, marcar la opción \opcion{Cola izquierda} y
% hacer click en el botón \boton{Aceptar}.
% \end{enumerate}
% \end{indication}
% 
% \item Suponiendo que la proporción real de hombres en la población fuese de $0.4$, ¿cuál es la potencia del contraste
% para la región de rechazo del apartado anterior?
% \begin{indication}
% \begin{enumerate}
% \item Select the menu \menu{Teaching > Distribuciones > Discretas > Binomial > Probabili\-dades acumuladas}.
% \item In the dialog displayed, introducir 1 en el campo \field{Valor(es) de la variable}, 10 en el campo
% \field{Nº de repeticiones}, $0.4$ en el campo \field{Probabilidad de éxito}, marcar la opción \opcion{Cola izquierda} y
% hacer click en el botón \boton{Aceptar}.
% \end{enumerate}
% \end{indication}
% 
% \item Si en lugar de una muestra de tamaño 10 se tomase una muestra de tamaño 100, y haciendo uso de la aproximación de
% una distribución binomial mediante una normal, ¿qué número de hombres en la muestra formarían la región de rechazo para
% un riesgo $\alpha=0.05$? 
% ¿Qué potencia tendría ahora el contraste si la proporción real de hombres fuese de $0.4$? 
% ¿Es mejor o peor contraste que el anterior? 
% Justificar la respuesta.
% \begin{indication}
% Una distribución binomial $B(100,\, 0.5)$ puede aproximarse mediante una normal $N(50,5)$.
% \begin{enumerate}
% \item Select the menu \menu{Teaching > Distribuciones > Continuas > Normal > Cuantiles}.
% \item In the dialog displayed, introducir las probabilidad $0.05$ en el campo \field{Probabilidades}, 50 en
% el campo \field{media}, 5 en el campo \field{desviación típica}, marcar la opción \opcion{Cola izquierda} y hacer click
% en el botón \boton{Aceptar}.
% \end{enumerate}
% El valor obtenido es la frontera entre la región de aceptación y la región de rechazo. 
% Si en la muestra se obtienen menos hombres de dicho valor se rechazará la hipótesis nula, mientras que si se obtienen
% más se aceptará. 
% Para calcular la potencia de contraste:
% \begin{enumerate}
% \item Select the menu \menu{Teaching > Distribuciones > Continuas > Normal > Probabili\-dades acumuladas}.
% \item In the dialog displayed, introducir el valor de la frontera en el campo \field{Valor(es) de la
% variable}, 40 en el campo \field{media}, $4.899$ en el campo \field{desviación típica}, marcar la opción \opcion{Cola
% izquierda} y hacer click en el botón \boton{Aceptar}.
% \end{enumerate}
% \end{indication}
% 
% \item Si se toma una muestra de tamaño 100 y se observan 41 hombres, ¿cuál es $p$-valor del contraste? 
% ¿Podría rechazarse la hipótesis nula pra un riesgo $\alpha=0.05$? 
% ¿y para un riesgo $\alpha=0.01$?
% \begin{indication}
% \begin{enumerate}
% \item Select the menu \menu{Teaching > Test paramétricos > Proporciones > Test para una proporción}.
% \item In the dialog displayed marcar la casilla de \opcion{Introducción manual de frecuencias}, introducir
% 41 en el campo \field{Frecuencia muestral} e introducir 100 en el campo \field{Tamaño muestral}.
% \item En la solapa \mtab{Test options}, introducir $0.5$ en el campo \field{Hipótesis nula}, seleccionar
% como hipótesis alternativa \opcion{Unilateral menor} y hacer click en el botón \boton{Aceptar}.
% \end{enumerate}
% \end{indication}
% \end{enumerate}
% 
\item  The active ingredient concentration of a random sample of 10 drug containers drawn from a batch are (in
mg/mm$^{3}$)
\[
17.6\quad 19.2\quad 21.3\quad 15.1\quad 17.6\quad 18.9\quad 16.2\quad 18.3\quad 19.0\quad 16.4
\]

Do the following operations:
\begin{enumerate}
\item Create a data set with the variable \variable{concentration} and enter the data of the sample.

\item Test the two-sided hypothesis: $H_0$: $\mu=18$ vs $H_1$: $\mu\neq18$ with a significance level $0.05$.
\begin{indication}
\begin{enumerate}
\item Select the menu \menu{Teaching > Parametric tests > Means > t-test for the mean of one population}.
\item In the dialog displayed insert the variable \variable{concentration} in the field \field{Mean of}.
\item In the \mtab{Test options} tab enter 18 in the field \field{Null hypothesis}, check the box \option{Two-sided}
for the alternative hypothesis and click the button \button{Submit}.
\end{enumerate}
\end{indication}

\item Test the two-sided hypothesis: $H_0$: $\mu=19.5$ vs $H_1$: $\mu\neq19.5$ with significance levels $0.05$ and
$0.01$.
How affects the significance level to the test?

\begin{indication}
\begin{enumerate}
\item Select the menu \menu{Teaching > Parametric tests > Means > t-test for the mean of one population}.
\item In the dialog displayed insert the variable \variable{concentration} in the field \field{Mean of}.
\item In the \mtab{Test options} tab enter 19.5 in the field \field{Null hypothesis}, check the box \option{Two-sided}
for the alternative hypothesis and click the button \button{Submit}.
\end{enumerate}
\end{indication}

\item Test the two-sided hypothesis: $H_0$: $\mu=17$ vs $H_1$: $\mu\neq 17$ with a significance level $0.05$.
Test also the one-sided hypothesis: $H_0$: $\mu=17$ vs $H_1$: $\mu>17$ with a significance level $0.05$.
What is the relation between the $p$-value of the two-sided and the one-sided tests?
\begin{indication}
\begin{enumerate}
\item Select the menu \menu{Teaching > Parametric tests > Means > t-test for the mean of one population}.
\item In the dialog displayed insert the variable \variable{concentration} in the field \field{Mean of}.
\item In the \mtab{Test options} tab enter 17 in the field \field{Null hypothesis}, check the box \option{Two-sided}
for the two-sided alternative hypothesis or \option{Greater} for the one-sided alternative hypothesis and click the
button \button{Submit}.
\end{enumerate}
\end{indication}

\item If the manufacturer of the drug affirms that has increased the active ingredient concentration with respect to
previous batches where the concentration was 17 mg/mm$^3$, can we believe it?
\item What is the sample size required to detect an increase in the concentration mean of $0.5$ mg/mm$^{3}$ with a
significance level $\alpha=0.05$ and a power $1-\beta=0.8$?
\begin{indication}
To compute the sample size is required to know the population standard deviation or an estimate of it. 
To compute an estimate of the population standard deviation:
\begin{enumerate}
\item Select the menu \menu{Teaching > Descriptive statistics > Statistics}.
\item In the dialog displayed enter the variable \variable{concentration} in the field \field{Variable}.
\item In the \mtab{Basic statistics} tab check the box \option{Corrected standard deviation} and click the button
\button{Submit}.
\end{enumerate}
To compute the sample size:
\begin{enumerate}
\item Select the menu \menu{Teaching > Parametric tests > Means > Sample size for the t-test}.
\item In the dialog displayed check the box \option{One population} for the type of test, check the box
\option{one-sided} for the alternative hypothesis, enter $0.5$ in the field \field{Difference between the means}, enter the corrected standard deviation in the field \field{Standard deviation}, enter the value $0.05$ in the field \field{Significance
level}, enter the value $0.8$ in the field \field{Power} and click the button \button{Submit}.
\end{enumerate}
\end{indication}
\end{enumerate}


\item In a survey performed by a university about the use of the library, a random sample of 34 students has been asked
whether they go to the library at least once a week.
The answers are shown below. 
\begin{center}
\begin{tabular}{lllllllllllllllll}
no & yes & no & no & no & yes & no & yes & yes & yes & yes & no & yes & no & yes & no & no \\
no & yes & yes & yes & no & no & yes & no & no & yes & yes & no & no & yes & no & yes & no \\
\end{tabular}
\end{center}

\begin{enumerate}
\item Create a data set with the variable \variable{answer} and enter the data of the sample.

\item Test if the percentage of students that uses the library at least once a week is greater than 40\%.
\begin{indication}
\begin{enumerate}
\item Select the menu \menu{Teaching > Parametric test > Proportions > Test for one proportion}.
\item In the dialog displayed insert the variable \variable{answer} in the field \field{Variable} and enter \texttt{yes}
in the field \field{Proportion of}.
\item In the \mtab{Test options} tab enter $0.4$ in the field \field{Null hypothesis}, check the box \option{Greater}
for the alternative hypothesis and click the button \button{Submit}.
\end{enumerate}
\end{indication}
\end{enumerate}

\item A study tries to determine if there is difference between the ages at which two populations of babies $A$ and
$B$ begin to walk by themselves.
For testing it a random sample of every population was draw and the ages in moths at which the babies began
to walk was recorded.
The data is shown in the following table.
\[
\begin{tabular}{lrrrrrrrrrrrr}
\toprule
A & 9.5 & 10.5 & 9.0 & 9.8 & 10.0 & 13.0 & 10.0 & 13.5 & 10.0 & 9.8\\
B & 12.5 & 9.5 & 13.5 & 13.8 & 12.0 & 13.8 & 12.5 & 9.5 & 12.0 & 13.5 & 12.0 & 12.0\\
\bottomrule
\end{tabular}
\]

\begin{enumerate}
\item Create a data set with the variables \variable{age} and \variable{population} and enter the data of the sample.

\item Test if there is difference between the age means at which babies begin to walk in both populations
with a significance level $0.05$.
\begin{indication}
\begin{enumerate}
\item Select the menu \menu{Teaching > Parametric tests > Means > t-test for comparing the means of two independent
populations}.
\item In the dialog displayed insert the \variable{age} in the field \field{Compare mean of} and the variable
\variable{population} in the field \field{According to}.
\item In the frame \field{Populations to compare} insert the value \variable{A} in the field \field{Compare} and the
value \variable{B} in the field \field{With}.
\end{enumerate}
There are differences between the means if the $p$-value is less than $0.05$.
\end{indication}
\end{enumerate}


\item Some researchers have observed a greater airways resistance in smokers than in non-smokers.
To test the hypothesis the percentage of tracheobronchial retention was measured in a sample of persons when they were
smokers an after one year of quitting.
The data is shown in the following table.
\begin{center}
\begin{tabular}{cc}
\multicolumn{2}{c}{Percentage of tracheobronchial retention} \\
\toprule
Smoking & One year after quitting\\
60.6 & 47.5 \\
12.0 & 13.3 \\
56.0 & 33.0 \\
75.2 & 55.2 \\
12.5 & 21.9 \\
29.7 & 27.9 \\
57.2 & 54.3 \\
62.7 & 13.9 \\
28.7 & 8.90 \\
66.0 & 46.1 \\
25.2 & 29.8 \\
40.1 & 36.2 \\
\hline
\end{tabular}
\end{center}

\begin{enumerate}
\item Create a data set with the variables \variable{before} and \variable{after} and enter the data of the sample.

\item Perform the test to confirm or reject the hypothesis of researchers.
\begin{indication}
\begin{enumerate}
\item Select the menu \menu{Teaching > Parametric tests > Means > t-test for comparing the means of two paired
populations}.
\item In the dialog displayed insert the variable \variable{before} in the field \field{Compare mean of population} and
the variable \variable{after} in the field \field{With mean of population}.
\item In the \mtab{Test options} check the box \option{Greater} for the alternative hypothesis and click the button
\button{Submit}.
\end{enumerate}
\end{indication}
\end{enumerate}


\item In a course there are two groups of students, one in the morning and the other in the afternoon.
In the morning group 55 students of 80 passed, while in the afternoon group 32 students of 90 passed.
Are there significant differences between the percentages of students that passed in the morning and in the afternoon?
Can we conclude that the timetable is the cause of the differences?
Justify the answer. 
\begin{indication}
\begin{enumerate}
\item Select the menu \menu{Teaching > Parametric tests > Proportions > Test for comparing two proportions}.
\item In the dialog displayed check the box \option{Manual entry of frequencies}, enter $55$ in the field \field{Sample
frequency 1}, enter 80 in the field \field{Sample size 1}, enter 32 in the field \field{Sample frequency 2} and enter 90
in the field \field{Sample size 2}.
\item In the \mtab{Test options} tab check the box \option{Two-sided} and click the button \button{Submit}.
\end{enumerate}
\end{indication}
\end{enumerate}


\section{Proposed exercises}
\begin{enumerate}[leftmargin=*] 
\item The data set \variable{neonates} of the package \variable{rk.Teaching}, contains information about the pulse of a
sample of people after doing different exercises: resting pulse in beats per minute (pulse1), pulse after doing
exercise in beats per minute (pulse2), type of exercise (ran, 1=running, 2=walking), gender (gender, 1=male, 2=female)
and the weight (weight).
Do the following operations:
\begin{enumerate}
\item Test if the resting pulse is less than 75 beats per minute.
\item What sample size is required to detect an increment of 2 beats per minute in the resting pulse with
significance level $0.05$ and power $0.9$?
\item Test if the pulse after running is greater than 85 beats per minute.
\item A person has slight tachycardia if the resting pulse is greater than 90 beats per minute.
Test if the percentage of people with slight tachycardia is greate than 5\%.
\item Can we conclude that exercise increases the pulse with significance level $0.05$?
And with significance level $0.01$?
Justify the answer.
\item Is there difference in the pulse means after walking and running?
Justify the answer.
\item Is there difference between the resting pulse means of males and females?
And between the pulse means after running?
Justify the answer.
\end{enumerate}

\end {enumerate}
