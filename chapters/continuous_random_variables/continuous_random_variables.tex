% Author: Alfredo Sánchez Alberca (asalber@ceu.es)

\chapter{Continuous Random Variables}

\section{Solved exercises}
\begin{enumerate}[leftmargin=*]
\item Suppose that a bus passes by a bus stop every 15 minutes and that a person can arrive at the bus stop at any moment with the same likelihood. 
Then, the variable that measures the waiting time for the bus follows an Uniform probability distribution model $U(0,15)$, since any waiting time between 0 and 15 minutes has the same likelihood of happening. 
\begin{enumerate}
\item Plot the graph of the density function of the waiting time. 
\begin{indication}
\begin{enumerate}
\item Select the menu \menu{Teaching > Distributions > Continuous > Uniform > Probability graph}.
\item In the dialog shown, enter \command{0} in the field \field{Minimum}, enter \command{15} in the field
\field{Maximum} and click the button \button{Submit}.
\end{enumerate}
\end{indication}

\item Plot the graph of the distribution function of the waiting time. 
\begin{indication}
\begin{enumerate}
\item Select the menu \menu{Teaching > Distributions > Continuous > Uniform > Probability graph}.
\item In the dialog shown, enter \command{0} in the field \field{Minimum}, enter \command{15} in the field
\field{Maximum}, check the box \option{Distribution function} and click the button \button{Submit}.
\end{enumerate}
\end{indication}

\item Compute the probability of waiting for the bus less than $5$ minutes.
\begin{indication}
\begin{enumerate}
\item Select the menu \menu{Teaching > Distributions > Continuous > Uniform > Probabilities}.
\item In the dialog shown, enter \command{5} in the field \field{Values of the variable}, enter \command{0} in the field \field{Minimum}, enter \command{15} in the field \field{Maximum} and click the button \button{Submit}.
\end{enumerate}
\end{indication}

\item Compute the probability of waiting for the bus more than $12$ minutes.
\begin{indication}
\begin{enumerate}
\item Select the menu \menu{Teaching > Distributions > Continuous > Uniform > Probabilities}.
\item In the dialog shown, enter \command{12} in the field \field{Values of the variable}, enter \command{0} in the field \field{Minimum}, enter \command{15} in the field \field{Maximum}, check the box \option{Upper} in the field \field{Accumulation tail} and click the button \button{Submit}.
\end{enumerate}
\end{indication}

\item Compute the probability of waiting for the bus between $5$ and $10$ minutes.
\begin{indication}
\begin{enumerate}
\item Select the menu \menu{Teaching > Distributions > Continuous > Uniform > Probabilities}.
\item In the dialog shown, enter \command{5, 10} in the field \field{Values of the variable}, enter \command{0} in the field \field{Minimum}, enter \command{15} in the field \field{Maximum} and click the button \button{Submit}.
\end{enumerate}
La probability $P(5\leq X\leq 10)$ is the difference between the probabilities obtained $P(X\leq 10)-P(X\leq 5)$.
\end{indication}

\item Compute the time such that half of the times the person have to wait for the bus less than that time.
\begin{indication}
\begin{enumerate}
\item Select the menu \menu{Teaching > Distributions > Continuous > Uniform  > Quantiles}.
\item In the dialog shown, enter \command{0.5} in the field \field{Cumulative probabilities},
enter \command{0} in the field \field{Minimum}, enter \command{15} in the field \field{Maximum} and click the button \button{Submit}.
\end{enumerate}
\end{indication}

\item Compute the time such that 10\% of the times the person have to wait for the bus more than that time.
\begin{indication}
\begin{enumerate}
\item Select the menu \menu{Teaching > Distributions > Continuous > Uniform  > Quantiles}.
\item In the dialog shown, enter \command{0.1} in the field \field{Cumulative probabilities},
enter \command{0} in the field \field{Minimum}, enter \command{15} in the field \field{Maximum}, check the box \option{Upper} in the field \field{Accumulation tail} and click the button \button{Submit}.
\end{enumerate}
\end{indication}
\end{enumerate}


\item The random variable following a Normal probability distribution model with mean 0 and standard deviation 1, $Z\sim N(0,1)$, it is known as the Standard Normal.  
\begin{enumerate}
\item Plot the graph of the density function of $Z$.
\begin{indication}
\begin{enumerate}
\item Select the menu \menu{Teaching > Distributions > Continuous > Normal > Probability graph}.
\item In the dialog shown, enter \command{0} in the field \field{Mean}, enter \command{1} in the field \field{Standard deviation} and click the button \button{Submit}.
\end{enumerate}
\end{indication}

\item How does affect the mean and the standard deviation to the shape of the Gauss bell?
\begin{indication}
\begin{enumerate}
\item Select the menu \menu{Teaching > Distributions > Continuous > Normal > Probability graph}.
\item In the dialog shown, check the box \option{Preview}.
\item Change the value of the mean and observe how changes the shape of the Gauss bell.
\item Then change the value of the standard deviations and observe how changes the shape of the Gauss bell.
\end{enumerate}
\end{indication}

\item Plot the graph of the distribution function of $Z$.
\begin{indication}
\begin{enumerate}
\item Select the menu \menu{Teaching > Distributions > Continuous > Normal > Probability graph}.
\item In the dialog shown, enter \command{0} in the field \field{Mean}, enter \command{1} in the field \field{Standard deviation}, check the box \option{Distribution function} and click the button \button{Submit}.
\end{enumerate}
\end{indication}

\item Compute the probability $P(Z<-1)$. 
\begin{indication}
\begin{enumerate}
\item Select the menu \menu{Teaching > Distributions > Continuous > Normal > Probabilities}.
\item In the dialog shown, enter \command{-1} in the field \field{Values of the variable}, enter \command{0} in the field \field{Mean}, enter \command{1} in the field \field{Standard deviation}, and click the button \button{Submit}.
\end{enumerate}
\end{indication}

\item Compute the probability $P(Z>1)$. 
\begin{indication}
\begin{enumerate}
\item Select the menu \menu{Teaching > Distributions > Continuous > Normal > Probabilities}.
\item In the dialog shown, enter \command{1} in the field \field{Values of the variable}, enter \command{0} in the field \field{Mean}, enter \command{1} in the field \field{Standard deviation}, check the box \option{Upper} in the field \field{Accumulation tail} and click the button \button{Submit}.
\end{enumerate}
\end{indication}

\item Compute the probability that $Z$ takes a value between the mean minus the standard deviation and the mean plus the standard deviation, that is, $P(-1\leq Z\leq 1)$. 
\begin{indication}
\begin{enumerate}
\item Select the menu \menu{Teaching > Distributions > Continuous > Normal > Probabilities}.
\item In the dialog shown, enter \command{-1, 1} in the field \field{Values of the variable}, enter \command{0} in the field \field{Mean}, enter \command{1} in the field \field{Standard deviation}, and click the button \button{Submit}.
\end{enumerate}
The probability $P(-1\leq Z\leq 1)$ is the difference between the probabilities obtained \mbox{$P(Z\leq 1)-P(Z\leq -1)$}.
\end{indication}

\item Compute the probability that $Z$ takes a value between the mean minus two times the standard deviation and the mean plus two times the standard deviation, that is, \mbox{$P(-2\leq Z\leq 2)$}. 
\begin{indication}
\begin{enumerate}
\item Select the menu \menu{Teaching > Distributions > Continuous > Normal > Probabilities}.
\item In the dialog shown, enter \command{-2, 2} in the field \field{Values of the variable}, enter \command{0} in the field \field{Mean}, enter \command{1} in the field \field{Standard deviation}, and click the button \button{Submit}.
\end{enumerate}
The probability $P(-2\leq Z\leq 2)$ is the difference between the probabilities obtained \mbox{$P(Z\leq 2)-P(Z\leq -2)$}.
\end{indication}

\item Compute the probability that $Z$ takes a value between the mean minus three times the standard deviation and the mean plus three times the standard deviation, that is, \mbox{$P(-3\leq Z\leq 3)$}. 
\begin{indication}
\begin{enumerate}
\item Select the menu \menu{Teaching > Distributions > Continuous > Normal > Probabilities}.
\item In the dialog shown, enter \command{-3, 3} in the field \field{Values of the variable}, enter \command{0} in the field \field{Mean}, enter \command{1} in the field \field{Standard deviation}, and click the button \button{Submit}.
\end{enumerate}
The probability $P(-3\leq Z\leq 3)$ is the difference between the probabilities obtained \mbox{$P(Z\leq 3)-P(Z\leq -3)$}.
\end{indication}

\item Compute the quartiles.
\begin{indication}
\begin{enumerate}
\item Select the menu \menu{Teaching > Distributions > Continuous > Normal > Quantiles}.
\item In the dialog shown, enter \command{0.25, 0.5, 0.75} in the field \field{Cumulative probabilities}, enter \command{0} in the field \field{Mean}, enter \command{1} in the field \field{Standard deviation} and click the button \button{Submit}.
\end{enumerate}
\end{indication}

\item Compute the value of the standard normal with a lower probability tail of $0.95$.
\begin{indication}
\begin{enumerate}
\item Select the menu \menu{Teaching > Distributions > Continuous > Normal > Quantiles}.
\item In the dialog shown, enter \command{0.95} in the field \field{Cumulative probabilities}, enter \command{0} in the field \field{Mean}, enter \command{1} in the field \field{Standard deviation} and click the button \button{Submit}.
\end{enumerate}
\end{indication}

\item Compute the value of the standard normal with an upper probability tail of $0.025$.
\begin{indication}
\begin{enumerate}
\item Select the menu \menu{Teaching > Distributions > Continuous > Normal > Quantiles}.
\item In the dialog shown, enter \command{0.025} in the field \field{Cumulative probabilities}, enter \command{0} in the field \field{Mean}, enter \command{1} in the field \field{Standard deviation}, check the box \option{Upper} in the field \field{Accumulation tail} and click the button \button{Submit}.
\end{enumerate}
\end{indication}
\end{enumerate}


% \item El teorema central del límite establece que the variable resultante de sumar 30 o más variables independientes
% sigue una distribución normal de media the suma de las medias de cada una de las variables and de varianza the suma de sus
% varianzas.
% Esta es the explicación de que una gran parte de las variables continuas que aparecen en the naturaleza sean variables
% normales.
% Para observar de manera experimental el teorema central del límite se realiza un experimento que consiste en lanzar
% varios dados muchas veces and sumar los valores obtenidos. 
% Se pide:
% \begin{enumerate}
% \item Simular el lanzamiento de un dado 100000 veces and dibujar el diagrama de barras asociado. 
% ¿Tiene forma de campana de Gauss?
% \begin{indication}Para generar los lanzamientos del dado: 
% \begin{enumerate}
% \item Select the menu \menu{Teaching >Simulaciones>Lanzamiento de dados}.
% \item In the dialog shown, enter 1 in the field \field{Número de dados}, enter 100000 en el
% field \field{Número de lanzamientos}, select the opción \option{Incluir suma}, enter un nombre para el conjunto
% de datos and click the button \button{Submit}.
% \end{enumerate}
% Para dibujar el diagrama de barras:
% \begin{enumerate}
% \item Select the menu \menu{Teaching>Gráficos>Diagrama de barras}.
% \item In the dialog shown select the variable \variable{sum}.
% \item En the solapa \menu{Opciones de las barras}, select the opción \option{Frequencies relativas} and hacer clic en
% el button \button{Submit}.
% \end{enumerate}}
% \end{indication}
% 
% \item Repetir el apartado anterior con 2 and 30 dados. 
% ¿Se cumple el teorema central del límite?
% \end{enumerate}


\item If $X_1,\ldots ,X_n$ are $n$ independent standard normal variables, then the variable $X=X_1^2 + \ldots + X_n^2$ follows a probability distribution model Chi-square with $n$ degrees of freedom $\chi^2(n)$. 
Let $X$ be a variable following a Chi-square probability distribution model with 6 degrees of freedom, $\chi^2(6)$. 
\begin{enumerate}
\item Plot the graph of the density function of $X$.
\begin{indication}
\begin{enumerate}
\item Select the menu \menu{Teaching > Distributions > Continuous > Chi-square > Probability graph}.
\item In the dialog shown, enter \command{6} in the field \field{Degrees of freedom} and click the button \button{Submit}.
\end{enumerate}
\end{indication}


\item Compute the probability $P(X<6)$.
\begin{indication}
\begin{enumerate}
\item Select the menu \menu{Teaching > Distributions > Continuous > Chi-square > Probabilities}.
\item In the dialog shown, enter \command{6} in the field \field{Values of the variable}, enter \command{6} in the field \field{Degrees of freedom} and click the button \button{Submit}.
\end{enumerate}
\end{indication}

\item Compute the 5th percentile.
\begin{indication}
\begin{enumerate}
\item Select the menu \menu{Teaching > Distributions > Continuous > Chi-square > Quantiles}.
\item In the dialog shown, enter \command{0.05} in the field \field{Cumulative probabilities}, enter \command{6} in the field \field{Degrees of freedom} and click the button \button{Submit}.
\end{enumerate}
\end{indication}

\item Compute el value with an upper probability tail of $0.1$.
\begin{indication}
\begin{enumerate}
\item Select the menu \menu{Teaching > Distributions > Continuous > Chi-square > Quantiles}.
\item In the dialog shown, enter \command{0.1} in the field \field{Cumulative probabilities}, enter \command{6} in the field \field{Degrees of freedom}, check the box \option{Upper} in the field \field{Accumulation tail} and click the button \button{Submit}.  
\end{enumerate}
\end{indication}
\end{enumerate}


\item If $Y$ is a Chi-square variable with $n$ degrees of freedom and $Z$ a standard normal variable independents, then the variable $X=\frac{Z}{\sqrt{Y/n}}$ follows a Student's t probability distribution model with $n$ degrees of freedom, $T(n)$. 
Let $X$ be a variable following a Student's t probability distribution model with 8 degrees of freedom, $T(8)$.

\begin{enumerate}
\item Plot the graph of the density function of $X$ and compare it with the standard normal one.
\begin{indication}
\begin{enumerate}
\item Select the menu \menu{Teaching > Distributions > Continuous > Student's t> Probability graph}.
\item In the dialog shown, enter \command{8} in the field \field{Degrees of freedom} and click the button \button{Submit}.
\end{enumerate}
\end{indication}

\item Compute the 8th percentile.
\begin{indication}
\begin{enumerate}
\item Select the menu \menu{Teaching > Distributions > Continuous > Student's t > Quantiles}.
\item In the dialog shown, enter \command{0.08} in the field \field{Cumulative probabilities}, enter \command{8} in the field \field{Degrees of freedom} and click the button \button{Submit}.
\end{enumerate}
\end{indication}

\item Compute el value such that 5\% of the population is above that value.
\begin{indication}
\begin{enumerate}
\item Select the menu \menu{Teaching > Distributions > Continuous > Student's t > Quantiles}.
\item In the dialog shown, enter \command{0.05} in the field \field{Cumulative probabilities}, enter \command{8} in the field \field{Degrees of freedom}, check the box \option{Upper} in the field \field{Accumulation tail} and click the button \button{Submit}.  
\end{enumerate}
\end{indication}
\end{enumerate}


\item If $Y_1$ and $Y_2$ are two independent Chi-square variables with $n$ and $m$ degrees of freedom respectively, then the variable
\[
X=\frac{Y_1/n}{Y_2/m}
\]
follows a Fishers' F probability distribution model with $n$ and $m$ degrees of freedom, $F(n,m)$. 
Let $X$ be a variable following a Fisher's F probability distribution model with $10$ and $20$ degrees of freedom, $F(10,20)$. 

\begin{enumerate}
\item Plot the graph of the density function of $X$.
\begin{indication}
\begin{enumerate}
\item Select the menu \menu{Teaching > Distributions > Continuous > Fishers' F > Probability graph}.
\item In the dialog shown, enter \command{10} in the field \field{Numerator degrees of freedom}, enter \command{20} in the field \field{Denominator degrees of freedom} and click the button \button{Submit}.
\end{enumerate}
\end{indication}

\item Compute the probability $P(X>1)$. 
\begin{indication}
\begin{enumerate}
\item Select the menu \menu{Teaching > Distributions > Continuous > Fishers' F > Probabilities}.
\item In the dialog shown, enter \command{1} in the field \field{Values of the variable}, enter \command{10} in the field \field{Numerator degrees of freedom}, enter \command{20} in the field \field{Denominator degrees of freedom}, check the box \option{Upper} in the field \field{Accumulation tail} and click the button \button{Submit}.
\end{enumerate}
\end{indication}

\item Compute the interquartile range.
\begin{indication}
\begin{enumerate}
\item Select the menu \menu{Teaching > Distributions > Continuous > Fishers' F > Quantiles}.
\item In the dialog shown, enter \command{0.25, 0.75} in the field \field{Cumulative probabilities}, enter \command{10} in the field \field{Numerator degrees of freedom}, enter \command{20} in the field \field{Denominator degrees of freedom} and click the button \button{Submit}.
\end{enumerate}
The interquartile range is the difference between the third and the first quartiles. 
\end{indication}
\end{enumerate}

\end{enumerate}


\section{Proposed exercises}
\begin{enumerate}[leftmargin=*]
\item It is known that the glucose level in blood of diabetic persons follows a normal distribution model with
mean 106 mg/100 ml and standard deviation 8 mg/100 ml.
\begin{enumerate}
\item Calculate the probability of a random diabetic person having a glucose level less than 120 mg/100 ml. 
\item What percentage of persons have a glucose level between 90 and 120 mg/100 ml?
\item Calculate and interpret the first quartile of the glucose level. 
\end{enumerate}

\item It is known that the cholesterol level in males 30 years old follows a normal distribution with mean 220 mg/dl and
standard deviation 30 mg/dl. 
If there are 20000 males 30 years old in the population,
\begin{enumerate}
\item how many of them have a cholesterol level between 210 and 240 mg/dl?
\item If a cholesterol level greater than 250 mg/dl can provoke a thrombosis, how many of them are in risk of
thrombosis?
\item Calculate the cholesterol level above which 20\% of the males are?
\end{enumerate}
\end{enumerate}
